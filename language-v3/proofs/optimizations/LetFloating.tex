\begin{lemma}[Float-in preserves types (and thus is sound)]
    \emph{Not trivial to define. Perhaps best done by case analysis}
\end{lemma}

\begin{proof}~

\begin{tabbing}

\end{tabbing}
\end{proof}


\begin{lemma}[Full-laziness preserves types (and thus is sound)]
    \emph{Not trivial to define. Perhaps best done by case analysis?}
\end{lemma}

\begin{proof}~

\begin{tabbing}

\end{tabbing}
\end{proof}



\begin{lemma}[Local-transformations preserve types]
There are three lemmas for local transformations:
  \[
  \begin{array}{llcl}
  1. & \G \vdash (\llet{v = e}{b})~a : \vp & \Longrightarrow & \G \vdash \llet{v = e}{b~a} : \vp\\
  2. & \G \vdash \ccase{(\llet{v = e}{b})}{\dots} : \vp & \Longrightarrow & \G \vdash \llet{v = e}{\ccase{b}{\dots}} : \vp\\
  3. & \G \vdash \llet{x = (\llet{v = e}{b})}{c} : \vp & \Longrightarrow & \G \vdash \llet{v = e}{\llet{x = b}{c}} : \vp\\
  \end{array}
  \]
\end{lemma}

\begin{description}
\item[1.] First local-transformation
\begin{proof}~

\begin{tabbing}

\end{tabbing}
\end{proof}

\item[2.] Second local-transformation
\begin{proof}~

\begin{tabbing}

\end{tabbing}
\end{proof}

\item[3.] Third local-transformation
\begin{proof}~

\begin{tabbing}

\end{tabbing}
\end{proof}
\end{description}




