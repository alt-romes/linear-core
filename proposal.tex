\documentclass[a4paper, draft]{article}

\usepackage{cmll}
\usepackage{amssymb}
\usepackage{amsmath}
\usepackage{mathpartir}
\usepackage[ruled,vlined]{algorithm2e}
\usepackage{hyperref}
\usepackage{fancyvrb}
\DefineVerbatimEnvironment{code}{Verbatim}{fontsize=\small}
\DefineVerbatimEnvironment{example}{Verbatim}{fontsize=\small}

\title{Linting Linearity in Core/System FC}
\author{Rodrigo Mesquita}


\begin{document}
\maketitle

\section{Introduction}

\subsection{Motivation}

\begin{itemize}
    \item Are we really preserving linearity?
    \item Inform/unlock other optimisations that take into account linearity
\end{itemize}


\section{Background}

\begin{itemize}
    \item Linear types
    \item Linear Haskell
    \item Core
    \item Sistemas de inferência
    \item GADTs e Coercions
\end{itemize}

\section{Related Work}


\section{Technical Details}

Since Linear Haskell's\ref{}
publication and implementation release in GHC 9.2, Haskell's type system
supports linearity annotations in functions -- bringing linear types into a
mainstream pure and lazy functional language.

System FC is the formal system in which the implementation of GHC's intermediate
representation language \emph{Core} is based on.

There are at least two distinct typecheckers in core. The first is run on the frontend
language, i.e. the Haskell we write, and is a big and complex typechecker. The
second is run on the intermediate language \emph{Core} that we obtain from
desugaring Haskell.

\emph{Core} is a much smaller and more principled language than the whole of
Haskell (even though we can compile the whole of Haskell to it), and the
typechecker for it is small and fast due to \emph{Core} being explicitly typed
and having a very small abstract syntax tree. This typechecker is called
\emph{Lint} and gives us guarantees of correctness (i.e. sanity) in face of
the complexity of all the transformations a Haskell program undergoes, such as
type inference, desugaring and optimising transformations (and other Core
related passes?).

\begin{itemize}
    \item In GHC, the type Type is shared between the surface language and Core.
        So modifying the types implies that Core is modified as well. 

    \item More importantly, because Core is a good check that our implementation
        works as intended. That is, a linearly typed Core will ensure that
        linearly-typed programs are indeed desugared to linearly-typed programs
        in Core. And that optimisations do not destroy linearity.
\end{itemize}

The addition of linear types comes at two levels: frontend and Core.
In the frontend, we can now annotate with linearity type declarations and
typecheck programs that use them accordingly (a linear fuction will consume its
argument exactly once if it is consumed exactly once, for some definition of
\emph{consume} that I should revise here). In Core, we still have linearity
annotations and ideally \emph{Lint} would check that all the GHC transformations
to our linear program preserved its linearity. However, \textbf{it can't}!

Despite the strong formal foundations of linear types driving the
implementation, their interaction with the whole of GHC is still far from
trivial. Diverse problems with linearity spring up when we get past the
desugarer. In particular, optimizing transformations, coercions from GADTs and
type families, recursive lets, and empty case expressions don't currently fit in
with linearity.

We believe that GHC's transformations are correct, and it is the linear type
system that can't accommodate the resulting programs. We aim to formalise a type
system that is able to type check \emph{Core}/\emph{System FC} at all points in
the core pipeline and implement it into GHC to be able to preserve linearity
accross the stages and to enable \emph{Lint} to preserve our sanity regarding
linearity.

% Our key innovation is that, by recognising join points as a language construct,
% we both preserve join points through subsequent transformations and exploit them
% to make those transformations more effective. Next, we formalize this ap-
% proach; subsequent sections develop the consequences.

\section{Typing Usage Environments}

The first set of problems appears in the core-to-core optimisation passes. GHC
applies many optimising transformations to \emph{Core} and we believe those
transformations preserve linearity. However, our linear type system cannot check
that they indeed preserve linearity.

The simpler examples come from straightforward and common optimising
transformations. Then we have recursive let definitions that don't accommodate
linearity even though it might converge to only use the value once. Finally, we
have the empty case expression introduced with the \emph{EmptyCase} language
extension that we currently can't typecheck either.


A usage environment is a mapping from variables to multiplicities.

The key idea is to annotate every variable with either its multiplicity, if it's
lambda bound, or with its usage environment, if it's let bound.

For example, if y and z are linear, in the following code it might not look as
if y and z were both being consumed linearly, but indeed they are since in the
first branch we use x which means using y and z linearly, and we use y and z
directly on the second branch. Note that let binding x doesn't consume y and z,
only using x itself does, because of laziness.
\begin{code}
let x = (y, z) in
case e of
  Pat1 -> … x …
  Pat2 -> … y … z …
\end{code}

If we annotate the x bound by let with a usage environment $\delta$ mapping all (free?)
variables in its binder to a multiplicity ($\delta = [y := 1, z := 1]$), we
could, upon finding x, simply emit that y and z are consumed once. When typing
the second branch we'd also see that y and z are used exactly once, hence the
linearity in the two branches match.

Currently, in GHC, we don't annotate let-bound variables with a usage
environment, but we already calculate a usage environment and use it to check
some things (which things?)

\begin{itemize}
    \item occurrences? store usage environment in Ids (vars)
    \item Recursive lets (can it be rewritten using fix)
    \item Empty case expression
\end{itemize}

What do we currently do?

When we lint a core expression, we get both its type and its usage environment.
That means that to lint linearity in an expression, whenever we come across a
free variable we compute its usage environment and take it into account

\subsection{Inlining}

If we annotate the let bound variables with their usage and emit that usage when
we come across those variables, we can solve the linearity issues with inlining.

\subsection{Recursive Lets}

Optimisations can create a letrec which uses a variable linearly. The following
example uses 'x' linearly, but this is not seen by the linter.
\begin{code}
letrec f = \case
        True -> f False
        False -> x
in f True
\end{code}

Following the idea of making let-bound variables remember the usage environment
here's an informal description that that example is well typed. We want to
anotate the let-bound variable $f$ with a usage environment delta $\delta$, and
use the binding body to compute it.

Now, how to compute the usage environment of a case expression? It's described
here if I'm not mistaken
https://gitlab.haskell.org/ghc/ghc/-/wikis/uploads/355cd9a03291a852a518b0cb42f960b4/minicore.pdf.

However, not understanding it, my take would be that we can compute the usage
environment of every branch of the case expression and make sure that they all
unify (wrt to submultiplicities). The special case is when the branch happens to
call $f$ itself while computing the usage environment of $f$. If it were another
let bound variable we'd add its own usage environment to the one we're
computing; if it were a lambda bound variable we'd add [itself $:=$ its
multiplicity].

My idea is that we can emit a special usage [$rec := p$], which, when unified
against the other case branches $\delta$ will always succeed with a unification
mapping from $rec \rightarrow p\delta$ scaled by the multiplicity of $rec$

So taking the example, to compute the usage environment of $f$, we'd compute for
the second branch $[x := 1]$ and for the first branch $[rec := 1]$. Then, we'd
unify them with $rec \rightarrow [x := 1*1]$, and somehow result in $[x := 1]$

An example which should break linearity because $x$ is not linear in the first
branch:
\begin{code}
letrec f = \case
         True -> f False + f False
         False -> x
    in f True
\end{code}

To compute the usage environment of $f$ we take the second branch usage
environment $[x := 1]$ and the first branch usage $[rec := 1+1]$ and unify them,
somehow resulting in $[x := 2]$. Now, to lint the linearity in the whole let
expression we must ensure that the body of the let uses $x$ linearly. The body
is $f True$, which is a let-bound variable (how to distinguish functions that
must be applied vs variables) so we take its usage environment $[x := 2]$ which
does not use $x$ linearly and thus breaks linearity.

\textbf{Re-explained:} to compute the usage environment of the recursive
let-bound function $f$ when applied, we compute $Z$ such that for all branch
alternatives $U,V,...$, $U \subset Z$, $V \subset Z$ and so on by tracking the
multiplicities and usage environments of variables that show up in the body and
by emitting a special keyword $rec$ everytime we find a saturated call of $f$
(how to handle unsaturated calls?) (how to have a more general solution that
doesn't require a keyword, perhaps something to do with fixed points); Then we
scale $Z \setminus \{(rec,_)\}$ by $Z[rec]$ and get $T$ which is the usage
environment of $f$ when saturated.

\textbf{Example 1 revisited:}
\begin{enumerate}
    \item Take $U = \{rec := 1\}$
    \item Take $V = \{x := 1\}$
    \item Take $Z = \{x := 1, rec := 1\}$
    \item Take $\pi = Z[rec] = 1$
    \item Take $W = Z \setminus \{rec\} = \{x := 1\}$
    \item Take $T = \pi W = \{x := \pi * 1\} = \{x := 1\}$
    \item Linearity OK
\end{enumerate}

\textbf{Example 2 revisited:}
\begin{enumerate}
    \item Take $U = \{rec := 2\}$
    \item Take $V = \{x := 1\}$
    \item Take $Z = \{x := 1, rec := 1\}$
    \item Take $\pi = Z[rec] = 2$
    \item Take $W = Z \setminus \{rec\} = \{x := 1\}$
    \item Take $T = \pi W = \{x := \pi * 1\} = \{x := 2\}$
    \item Linearity not OK
\end{enumerate}


Draft typing rule:


\begin{mathparpagebreakable}
    \infer*[right=(letrec)]
    {\Gamma ; x_1 : A_1 \dots x_n : A_n \vdash t_i : A_i \leadsto \{U_{i_\text{naive}}\} \\
     (U_1 \dots U_n) = computeRecUsages(U_{1_\text{naive}} \dots U_{n_\text{naive}}) \\
     \Gamma ; x_1 :_{U_1} A_1 \dots x_n :_{U_n} A_n \vdash u : C \leadsto \{V\}}
    {\Gamma \vdash \text{let } x_1 :_{U_1} A_1 = t_1 \dots x_n :_{U_n} A_n = t_n \text{ in } u : C \leadsto \{V\}}
\end{mathparpagebreakable}


The difficulty of calculating a usage environment for the recursive let recs
increases with the number of recursive definitions in the same let.

We devise an algorithm for computing the usage environment of a set of recursive
definitions. The proof that the algorithm works follows...

\begin{itemize}
    \item Given a list of functions and their naive environment computed from
        the body and including the recursive function names ($(f, F), (g, G),
        (h, H)$ in which there might exist multiple occurrences of $f, g, h$ in $F, G, H$
    \item For each bound rec var and its environment, update all bindings and
        their usage as described in the algorithm
    \item After iterating through all bound rec vars, all usage environments
        will be free of recursive bind usages, and hence "final"
\end{itemize}

-- I should probably use the re-computed usageEnvs instead of the
naiveUsageEnvs. I'm pretty sure it might fail in some inputs if I keep using the
naiveUsageEnvs.
\begin{algorithm}
$usageEnvs \gets naiveUsageEnvs.map(fst)$\;
\For{$(bind, U) \in naiveUsageEnvs$}{
    \For{$V \in usageEnvs$}{
        $V \gets sup(V[bind]*U\setminus\{bind\}, V\setminus\{bind\})$
    }
}
\caption{computeRecUsages}
\end{algorithm}

In haskell that would look like this;
\begin{code}
computeRecUsageEnvs :: [(Var, UsageEnv)] -- Recursive usage environments associated to their recursive call
                    -> [(Var, UsageEnv)] -- Non-recursive usage environments
computeRecUsageEnvs l =
  foldl (flip \\(v,vEnv) -> map \\(n, nEnv) -> (n, ((fromMaybe 0 \$ v `M.lookup` nEnv) `scale` (M.delete v vEnv)) `sup` (M.delete v nEnv))) l l

sup :: UsageEnv -> UsageEnv -> UsageEnv
sup = M.merge M.preserveMissing M.preserveMissing (M.zipWithMatched \$ \_ x y -> max x y)

scale :: Mult -> UsageEnv -> UsageEnv
scale m = M.map (*m)
\end{code}

\subsection{Handling the case binder}

The current typing rule for case expressions describes that using the case
binder is as using the case scrutinee multiple times (though it seems like it
actually just starts counting after the second usage which would make the rule
wrong for a case expression that doesn't use the binder -- the scrutinee should
also be consumed), so we scale the usage of
the case scrutinee by the superior multiplicity of using the case binder across
all branches. However, we hypothetize that this prevents us from typing many
valid programs and ultimately doesn't capture correctly the usage of variable,
and present a seemingly viable alternative.

Firstly, we remember the definition of \emph{consuming a value} from Linear
Haskell as
\begin{itemize}
    \item consuming an atomic value is forcing it to Normal Form (NF)
    \item consuming a value that is constructed with more than 1 argument is
        consuming all of its arguments
\end{itemize}
and further note that when an expression is scrutinized by a case expression it
is evaluated to Weak Head Normal Form (WHNF), which in the case of a nullary data
constructor is equal to NF.

Now, with the following example, my interpretation of the current rule says that
the following case expression consumes the scrutinee twice because the case
binder is used. However, we know for sure that in the branch where z is used, z
is equal to False, and using False doesn't violate linearity since we can
unrestrictedly create nullary data constructors. A situation similar to this one
below happens after the CSE transformation.
\begin{code}
    case <complex expression> $\leadsto U$ of z {
        False -> case y of z' { DEFAULT -> z }
        True  -> y
    }

A different example is using the case binder $z$ instead of the arguments we
pattern matched on $x,y$. This currently violates linearity because both $x$ and
$y$ aren't used and because $z$ is used twice. This doesn't typecheck in the
frontend either. However, we know that in this branch, using $z$ is effectively
the same as using $(x,y)$!
\begin{code}
    case <complex expression> $\leadsto U$ of z {
        (x,y) -> z
    }
\end{code}

The good news is both these two programs and a handful of others listed in
Section~\ref{examples} are accepted with a new rule we've devised for the
linear linter. We still need to prove we don't accept any invalid programs as
valid.

The key idea of the case-binder-usage solution is annotating the case binder
with independent usage environments for each pattern match. That is, for
each of the possible branches, using the case binder $z$ will equate to
using its usage environment in that branch. That makes it so that using the
case binder instead of a nullary data constructor is the same, using the case
binder instead of reconstructing the value with the bound pattern variables too,
and other situations in which we previously couldn't typecheck linearity are
now accepted.

The typing rule for the case expression using the case binder solution:

\begin{mathparpagebreakable}
    \infer*[right=(case)]
    {\Gamma \vdash t : D_{\pi_1 \dots \pi_n} \leadsto \{U\} \\
     \Gamma ; z :_{U_k} D_{\pi_1 \dots \pi_n} \vdash b_k : C \leadsto \{V_k\}
     \and V_k \leq V}
    {\Gamma \vdash \text{case } t \text{ of } z :_{(U_1\dots U_n)} D_{\pi_1 \dots \pi_n} \{b_k\}_1^m : C \leadsto \{U + V\}}
\end{mathparpagebreakable}


\subsection{Empty Case}

For case expressions, the usage environment is computed by checking all branches
and taking sup. However, this trick doesn't work when there are no branches.

\begin{itemize}
\item https://gitlab.haskell.org/ghc/ghc/-/issues/20058
\item https://gitlab.haskell.org/ghc/ghc/-/issues/18768

\item (1) Just like a case expression remembers its type (Note [Why does Case have a
'Type' field?] in Core.hs), it should remember its usage environment. This data
should be verified by Lint.

\item (2) Once this is done, we can remove the Bottom usage and the second field of
UsageEnv. In this step, we have to infer the correct usage environment for empty
case in the typechecker.
\end{itemize}

\begin{code}
{-# LANGUAGE LinearTypes, EmptyCase #-}
module M where

{-# NOINLINE f #-}
f :: a %1-> ()
f x = case () of {}
\end{code}

This example is well typed: a function is linear if it consumes its argument
exactly once when it's consumed exactly once. It seems like the function isn't
linear since it won't consume x because of the empty case, however, that also
means f won't be consumed due to the same empty case, thus linearity is
preserved.

\begin{code}
* In the case of empty types (see Note [Bottoming expressions]), say
  data T
we do NOT want to replace
  case (x::T) of Bool {}   -->   error Bool "Inaccessible case"
because x might raise an exception, and *that*'s what we want to see!
(#6067 is an example.) To preserve semantics we'd have to say
   x `seq` error Bool "Inaccessible case"
but the 'seq' is just such a case, so we are back to square 1.
\end{code}

There are three different problems:

\begin{itemize}
\item castBottomExpr converts (case x :: T of {}) :: T to x.
\item Worker/wrapper moves the empty case to a separate binding
\item CorePrep eliminates empty case, just like point 1 (See -- Eliminate empty
    case in GHC.CoreToStg.Prep
\end{itemize}

With castBottomExpr, we get the example above to
\begin{code}
    f = \ @a (x (%1) :: a) -> ()
\end{code}
And if we don't, 
\begin{code}
    f = \ @a (x (%1) :: a) -> case () of {}
\end{code}
And that supposedly if we had a usage environment in the case expression we
could avoid the error. How is it typed without the transformation in face
of the bottom? (Even knowing that theoretically it's because of divergence?)



\section{Multiplicity Coercions}

The second set of problems arises from our inability to coerce a multiplicity
into another (or say that one is submultiple of another?).

When we pattern match on a GADT we ...

Taking this example (copy example over) we can see that we don't know how to say
that x is indeed linear in one case and unrestricted in the other, even though
it is according to its type. We'd need some sort of coercion to coerce through
the multiplicity to the new one we uncover when we pattern match on the GADT
evidence (...)

\begin{code}
data SBool :: Bool -> Type where
  STrue :: SBool True
  SFalse :: SBool False

type family If b t f where
  If True t _ = t
  If False _ f = f

dep :: SBool b -> Int %(If b One Many) -> Int
dep STrue x = x
dep SFalse _ = 0
\end{code}


% \section{Novel rules}

% \subsection{The usage environment in a Recursive Let}

% \subsection{The usage environment of a Case Binder\label{caseBinderKey}}


% \begin{mathparpagebreakable}
%     \infer*[right=(letrec)]
%     {\Gamma ; \Delta/ \Delta' ; \Omega \vdash M : A \Uparrow \and \Gamma ;
%     \Delta/ \Delta'' ; \Omega \vdash N : B \Uparrow \and \Delta' = \Delta''}
%     {\Gamma ; \Delta/\Delta' ; \Omega \vdash  (M \with N) : A \leadsto U}
% \end{mathparpagebreakable}


\section{Examples\label{examples}}

In this section we present worked examples of programs that currently fail to
compile with \textbf{-dlinear-core-lint} but would succeed according to the
novel typing rules we introduced above. Each example belongs to a subsection
that indicates after which transformation the linting failed.

\subsection{After the desugarer (before optimizations)}

The definition for $\$!$ in \textbf{linear-base}\cite{} fails to lint after the
desugarer (before any optimisation takes place) with \emph{Linearity failure in
lambda: x 'Many $\not\subseteq$ p}
\begin{code}
($!) :: forall {rep} a (b :: TYPE rep) p q. (a %p -> b) %q -> a %p -> b
($!) f !x = f x
\end{code}

The desugared version of that function follows below. It violates (naive?)
linearity by using $x$ twice, once to force the value and a second time to call
$f$. However, the $x$ passed as an argument is actually the case binder and it
must be handled in its own way. The case binder is the key (as seen in ~\ref{caseBinderKey}) to solving this
and many other examples.
\begin{code}
($!)
  :: forall {rep :: RuntimeRep} a (b :: TYPE rep) (p :: Multiplicity)
            (q :: Multiplicity).
     (a %p -> b) %q -> a %p -> b
($!)
  = \ (@(rep :: RuntimeRep))
      (@a)
      (@(b :: TYPE rep))
      (@(p :: Multiplicity))
      (@(q :: Multiplicity))
      (f :: a %p -> b)
      (x :: a) ->
      case x of x { __DEFAULT -> f x }
\end{code}

% Intuitively, this program is linear because forcing the polymorphic value to
% WHNF ...? Need better semantics of consuming
%
The case binder usage rule typechecks this program because $x$ is consumed once,
the usage environment of the case binder is empty ($x :_\emptyset a$) and,
therefore, when $x$ is used in $f(x)$, we use its usage environment which is
empty, so we don't use anything new.

\emph{How are strictness annotations typed in the frontend? The issue is this program
consumes to value once to force it, and then again to determine the return
value.}

% Work better the meaning of consumption (does it mean for a value to be reduced
% to WHNF?). Is consuming = forcing? Why is the above program linear?

As a finishing note on this example, we show the resulting program from
\textbf{-ddump-simpl} that simply uses a different name for the case binder.
\begin{code}
($!)
  :: forall {rep :: RuntimeRep} a (b :: TYPE rep)
            (p :: Multiplicity) (q :: Multiplicity).
     (a %p -> b) %q -> a %p -> b
($!)
  = \ (@(rep :: RuntimeRep))
      (@a)
      (@(b :: TYPE rep_aSJ))
      (@(p :: Multiplicity))
      (@(q :: Multiplicity))
      (f :: a %p -> b)
      (x :: a) ->
      case x of y { __DEFAULT -> f_aDV y }
\end{code}
% So in this program we use another name for the case binder, but it still stands
% for the resulting of evaluating to WHNF; how is it technically different?

\subsection{Common Sub-expression Elimination}

Currently, the CSE seems to transform a linear program that pattern matches on
constant and returns the same constant into a program that breaks linearity that
pattern matches on the argument and returns the argument (where in the frontend
a constant equal to the argument would be returned)

The definition for $\&\&$ in \textbf{linear-base} fails to lint after the common
sub-expression elimination (CSE) pass transforms the program.
\begin{code}
(&&) :: Bool %1 -> Bool %1 -> Bool
False && False = False
False && True = False
True && x = x
\end{code}
The issue with the program resulting from the transformation is $x$ being used
twice in the first branch of the first case expression. We pattern match on y to
force it (since it's a type without constructor arguments, forcing is consuming)
and then return $x$ rather than the constant $False$
\begin{code}
(&&) :: Bool %1 -> Bool %1 -> Bool
(&&) = \ (x :: Bool) (y :: Bool) ->
  case x of w1 {
    False -> case y of w2 { __DEFAULT -> x };
    True -> y
  }
\end{code}
At a first glance, the resulting program is impossible to typecheck linearly.

The key observation is that after $x$ is forced into either $True$ or $False$,
we know $x$ to be a constructor without arguments (which can be created
without restrictions) and know that where we see $x$ we can as well have
$True$ or $False$ depending on the branch. However, using $x$ here is very
unsatisfactory (and linearly unsound?) because $x$ might be an expression, and
we can't really associate $x$ to the value we pattern matched on (be it $True$
or $False$). What we really want to have instead of $x$ is the $w1$ case binder --
it relates directly to the value we pattern matched on, it's a variable rather
than an expression, and, most importantly, our case-binder-usage idea could be
applied here as well

% Idea: if $x$ was anotated with some information regarding the CSE then perhaps
% it could be typechecked (in a system that considered said anotations) -- This is
% too specific and the case binder environment solution is much cleaner. It just
% requires that the example is changed into using the case binder $w1$ rather than
% $x$.

The solution with the unifying case binder usage idea means annotating the case
binder with its usage environment. For $True$ and $False$ (and other data
constructors without arguments) the usage environment is always empty (using
them doesn't entail using any other variable), meaning we can always use the
case binder instead of the actual constant constructor without issues.
%
In concrete, if we had $w1$ instead of the second occurrence of $x$, we'd have
an empty usage environment for $w1$ in the $False$ branch ($w1 :_\emptyset
Bool$) and upon using $w1$ we wouldn't use any extra resources


To make this work, we'd have to look at the current transformations and see
whether it would be possible to ensure that CSEing the case scrutinee would
entail using the case binder instead of the scrutinee. I don't know of a
situation in which we'd really want the scrutinee rather than the case binder,
so I hypothetize the modified transformation would work out in practice and
solve this linearity issue.


% In this exact example, the binder has usage environment empty ($w1 := []$),
% meaning that in the False branch, when we use $x$, if we instead used $w1$ which
% is equivalent and seems more correct (since it doesn't need the case scrutinee
% expression to be a variable), then the usage of $w1$ would imply the usage of
% $[]$ which is nothing and therefore we would preserve linearity

Curiously, the optimised program resulting from running all transformations
actually does what we expected it with regard to using the constant constructors
and preserving linearity. So is the issue from running linear lint after a
particular CSE but it would be fine in the end?
\begin{code}
(&&) :: Bool %1 -> Bool %1 -> Bool
(&&) = \ (x :: Bool) (y :: Bool) ->
  case x of {
    False -> case y of { __DEFAULT -> GHC.Types.False };
    True -> y
  }
\end{code}

\subsection{Compiling ghc with -dlinear-core-lint}

From the definition of groupBy in GHC.Data.List.Infinite
\begin{code}
groupBy :: (a -> a -> Bool) -> Infinite a -> Infinite (NonEmpty a)
groupBy eq = go
  where
    go (Inf a as) = Inf (a:|bs) (go cs)
      where (bs, cs) = span (eq a) as
\end{code}
we get the following core which violates linearity.
\begin{code}
groupBy = \ (@a) (eq :: a -> a -> Bool) (eta :: Infinite a) ->
  letrec {
    go :: Infinite a -> Infinite (NonEmpty a)
    go = \ (inf :: Infinite a) -> case inf of {
      Inf x xs -> let {
        ds :: ([a], Infinite a)
        ds = let {
            parteq :: a -> Bool
            parteq = eq x
        } in
          letrec {
            goZ :: Infinite a -> ([a], Infinite a)
            goZ = \ (inf' :: Infinite a) -> case wgo inf' of { (# wA, wB #) -> (wA, wB) };

            wgo :: Infinite a -> (# [a], Infinite a #)
            wgo = \ (inf' :: Infinite a) -> case inf' of wildX2 {
              Inf y ys ->
                join {
                    jp :: ([a], Infinite a) \%1 -> (# [a], Infinite a #)
                    jp (wwj :: ([a], Infinite a)) =
                        case wwj of wwj {
                            DEFAULT -> case wwj of { (wA, wB) -> (# wA, wB #) } }
                    } in
                      case parteq y of {
                        False ->
                            let {
                                ww :: ([a], Infinite a)
                                ww = ([] @a, wildX2)
                            } in jump jp ww;
                        True ->
                            let {
                                dy :: ([a], Infinite a)
                                dy = case wgo ys of { (# wA, wB #) -> (wA, wB)
                                }
                            } in
                                let {
                                    wwB :: [a]
                                    wwB = case dy of { (bs, cs) -> bs }
                                } in
                                    let {
                                        wwA :: [a]
                                        wwA = : @a y wwB
                                    } in let {
                                        wwC :: Infinite a
                                        wwC = case dy of { (bs, cs) -> cs }
                                    } in let {
                                        ww :: ([a], Infinite a)
                                        ww = (wwA, wwC)
                                    } in jump jp ww
                                }
              };
          } in
              case wgo xs of { (# wA, wB #) -> (wA, wB) }
      } in
          Inf @(NonEmpty a) (:| @a x (case ds of { (bs, cs) -> bs })) (case ds of { (bs, cs) -> go cs })
    };
  } in go eta
\end{code}
The issue is in the linear function that shows up in the core output (how does
the linearity end up there?). The wwj variable is used once in the case
expression, bound as the case binder, and used in the case body once again. Why
do we do that instead of pattern matching right away? Seems a bit redundant.

Observation: If the branch is DEFAULT, then the case binder binds the case
scrutinee which was just forced?, but hasn't been actually consumed, because we
haven't consumed its components. As long as the same branch doesn't consume the
pattern matching result and the case binder at the same time it should be fine?

In this example, we force wwj with the case binder, but we don't really
consume it (more precise definition of consume...), so we can use the case
binder meaning we're simply using x for the first time. Needs to be
formalised....

\begin{code}
jp :: ([a], Infinite a) \%1 -> (# [a], Infinite a #)
jp (wwj :: ([a], Infinite a)) =
    case wwj of wwj {
        DEFAULT -> case wwj of { (wA, wB) -> (# wA, wB #) }
    }
\end{code}

Solution with the case binder idea that unifies a different problems:
The case binder is annotated with a usage environment $U_wwj = []$, and its
usage is equal to using the usage environment

Here it would mean that second \textbf{case of} $wwj$ doesn't actually use any
variables and hence it isn't used twice.


\subsection{In Result of TcGblEnv axioms}

I don't understand this one yet. I would guess it has to do with multiplicity
coercions. Incorrect incompatible branch: CoAxBranch (src/Prelude/Linear/GenericUtil.hs:112:3-51):
\begin{code}
type family Fixup (f :: Type -> Type) (g :: Type -> Type) :: Type -> Type where
  Fixup (D1 c f) (D1 _c g) = D1 c (Fixup f g)
  Fixup (C1 c f) (C1 _c g) = C1 c (Fixup f g)
  Fixup (S1 c f) (S1 _c (MP1 m f)) = S1 c (MP1 m f) -- error in this constructor
  Fixup (S1 c f) (S1 _c f) = S1 c f
  Fixup (f :*: g) (f' :*: g') = Fixup f f' :*: Fixup g g'
  Fixup (f :+: g) (f' :+: g') = Fixup f f' :+: Fixup g g'
  Fixup V1 V1 = V1
  Fixup _ _ = TypeError ('Text "FixupMetaData: representations do not match.")
\end{code}


\subsection{Artificial examples}

Following the difficulties in consuming or not consuming the case binder and its
associated bound variables linearly we constructed some additional examples that
bring out the problem quite clearly
\begin{code}
case x of w1
    (x1,x2) -> case y:Bool of
                DEFAULT -> (x1,x2)
\end{code}
Is equal to (note that the case binder is being used rather than the $x$ case
scrutinee which could be an arbitrary expression and hence really cannot be
consumed multiple times?)
\begin{code}
case x of w1
    (x1,x2) -> case y:Bool of
                DEFAULT -> w1
\end{code}

We initially wondered whether x was being consumed if the pattern match ignored
some of its variables. We put that if it does have wildcards then it isn't
consuming x fully
\begin{code}
case x of w1
    (_,x2) -> Is x being consumed? no because not all of its components are
    being consumed?
\end{code}

Can we have a solution that even handles weird cases in which we pattern match
twice on the same expression but only on one constructor argument per time?
\begin{code}
case exp of w1
    (x1,_) -> case w1 of w2
                (_, x2) -> (x1, x2)
\end{code}

A double let rec in which both use a linear bound variable $y$
\begin{code}
letrec f = \case
        True  -> y
        False -> g True
       g = \case
        True -> f True
        False -> y
    in f False
\end{code}

We have to compute the usage environment of f and g.
For f, we have the first branch with usage environment $y$ and the second with
usage $g$, meaning we have $F = \{g := 1, y := 1\}$
For g, we have $G = \{f := 1, y := 1\}$. To calculate the usage environment of
$f$ to emit upon $f False$, we calculate ...

Refer to the algorithm for computing recursive environment


\subsection{Join Points}

When duplicating a case (in the case-of-case transformation), to avoid code
explosion, the branches of the case are first made into join points

\begin{code}
case e of
  Pat1 -> u
  Pat2 -> v
~~>
let j1 = u in
let j2 = v in
case e of
  Pat1 -> j1
  Pat2 -> j2
\end{code}

If there is any linear variable in u and v, then the standard
let rule above will fail (since j1 occurs only in one branch, and
so does j2).

However, if j1 and j2 were annotated with their usage environment,


\end{document}


