\documentclass{article}

\title{Linting Linearity in Core/System FC}
\author{Rodrigo Mesquita}


\begin{document}

\maketitle

\section{Introduction}

Since Linear Haskell's\ref{}
publication and implementation release in GHC 9.2, Haskell's type system
supports linearity annotations in functions -- bringing linear types into a
mainstream pure and lazy functional language.

System FC is the formal system in which the implementation of GHC's intermediate
representation language \emph{Core} is based on.

There are at least two distinct typecheckers in core. The first is run on the frontend
language, i.e. the Haskell we write, and is a big and complex typechecker. The
second is run on the intermediate language \emph{Core} that we obtain from
desugaring Haskell.

\emph{Core} is a much smaller and more principled language than the whole of
Haskell (even though we can compile the whole of Haskell to it), and the
typechecker for it is small and fast due to \emph{Core} being explicitly typed
and having a very small abstract syntax tree. This typechecker is called
\emph{Lint} and gives us guarantees of correctness (i.e. sanity) in face of
the complexity of all the transformations a Haskell program undergoes, such as
type inference, desugaring and optimising transformations (and other Core
related passes?).

The addition of linear types comes at two levels: frontend and Core.
What's the first and foremost reason for permeating Core with linearity? I can't
phrase it correctly, something wrt to the foundational formal design and system FC.
In the frontend, we can now annotate with linearity type declarations and
typecheck programs that use them accordingly (a linear fuction will consume its
argument exactly once if it is consumed exactly once, for some definition of
\emph{consume} that I should revise here). In Core, we still have linearity
annotations and ideally \emph{Lint} would check that all the GHC transformations
to our linear program preserved its linearity. However, \textbf{it can't}!

Despite the strong formal foundations of linear types driving the
implementation, their interaction with the whole of GHC is still far from
trivial. Diverse problems with linearity spring up when we get past the
desugarer. In particular, optimizing transformations, coercions from GADTs and
type families, recursive lets, and empty case expressions don't currently fit in
with linearity.

We believe that GHC's transformations are correct, and it is the linear type
system that can't accommodate the resulting programs. We aim to formalise a type
system that is able to type check \emph{Core}/\emph{System FC} at all points in
the core pipeline and implement it into GHC to be able to preserve linearity
accross the stages and to enable \emph{Lint} to preserve our sanity regarding
linearity.

\section{Motivation}

\begin{itemize}
    \item Are we really preserving linearity?
    \item Inform/unlock other optimisations that take into account linearity
\end{itemize}

\section{Typing Usage Environments}

The first set of problems appears in the core-to-core optimisation passes. GHC
applies many optimising transformations to \emph{Core} and we believe those
transformations preserve linearity. However, our linear type system cannot check
that they indeed preserve linearity.

The simpler examples come from straightforward and common optimising
transformations. Then we have recursive let definitions that don't accommodate
linearity even though it might converge to only use the value once. Finally, we
have the empty case expression introduced with the \emph{EmptyCase} language
extension that we currently can't typecheck either.


\begin{itemize}
    \item Transformations, occurrences?
    \item Recursive lets
    \item Empty case expression
\end{itemize}


\section{Multiplicity Coercions}

The second set of problems arises from our inability to coerce a multiplicity
into another (or say that one is submultiple of another?).

When we pattern match on a GADT we ...

Taking this example (copy example over) we can see that we don't know how to say
that x is indeed linear in one case and unrestricted in the other, even though
it is according to its type. We'd need some sort of coercion to coerce through
the multiplicity to the new one we uncover when we pattern match on the GADT
evidence (...)

\begin{itemize}
    \item Are we really preserving linearity?
    \item Inform/unlock other optimisations that take into account linearity
\end{itemize}

\end{document}



