
\begin{theorem}[Type preservation]
\emph{If $\Gamma \vdash e : \sigma$ and $e \longrightarrow e'$ then $\Gamma \vdash e' : \sigma$}
\end{theorem}

\begin{proof}
By structural induction on the small-step reduction.

%   \begin{description}
%   \item[Case:] $\imp$-{\sc param}
%     \begin{tabbing}
%       $\Delta \vdash \alpha \imp \alpha$ \` this case\\
%       $\TERASE{\Delta}{\alpha} = t_I$, with $\bounds_\Delta(\alpha) =
%       t_I(\ov{\tau})$ \` by definition\\
%       $t_I \imp t_I$ \` by rule $\imp_I$
%     \end{tabbing}
%
%   \item[Case:] ...

\begin{description}

\item[Case:] $(\lambda x{:}_\pi\sigma.~e)~e' \longrightarrow e[e'/x]$
\begin{tabbing}
    (1) $\Gamma, \Gamma' \vdash (\lambda x{:}_\pi\sigma.~e)~e' : \varphi$\\
    (2) $\Gamma \vdash (\lambda x{:}_\pi\sigma.~e) : \sigma\to_\pi\varphi$ \` by inversion on ($\lambda E$) \\
    (3) $\Gamma' \vdash e' : \sigma$ \` by inversion on ($\lambda E$) \\
    Subcase $\pi = 1,p$:\\
    (4) $\Gamma, x{:}_{1,p}\sigma \vdash e : \varphi$ \` by inversion on ($\lambda I$) \\
    (5) $\Gamma[\lin{\G'}/x], \Gamma' \vdash e[e'/x] : \varphi$ \` by linear subst. lemma (3,4) \\
    (6) $\Gamma[\lin{\G'}/x] = \G$ \` since $\G$ is well defined before $x$'s binding (1)\\
    Subcase $\pi = \omega$:\\
    (4) $\hasnolinearvars{\Gamma'}$ \` by inversion on ($\lambda E$) \\
    (5) $\G, \xo \vdash e : \vp$ \` by inversion on ($\lambda I$)\\
    (6) $\G, \G' \vdash e[e'/x] : \vp$ \` by unrestricted subst. lemma (3,4,5)\\
\end{tabbing}

\item[Case:] $(\Lambda p.~e)~\pi \longrightarrow e[\pi/p]$
\begin{tabbing}
(1) $\Gamma \vdash (\Lambda p.~e)~\pi : \sigma[\pi/p]$\\
(2) $\Gamma \vdash (\Lambda p.~e) : \forall p.~\sigma$ \` by inversion on ($\Lambda E$) \\
(3) $\Gamma \vdash_{mult} \pi$ \` by inversion on ($\Lambda E$) \\
(4) $\Gamma, p \vdash e : \sigma$ \` by inversion on ($\Lambda I$) \\
% (4) $p \notin \Gamma$ \` by inversion on ($\Lambda I$) \\
(5) $\Gamma \vdash e[\pi/p]:\sigma[\pi/p]$ \` by mult. subst. lemma (3,4) \\
\end{tabbing}

\item[Case:] $\llet{x{:}_\Delta\sigma = e}{e'}\longrightarrow e'[e/x]$
\begin{tabbing}
(1) $\G, \G', \D\vdash \llet{x{:}_\Delta\sigma = e}{e'} : \varphi$\\
(2) $\G, \D, \xD \vdash e' : \varphi$ \` by inversion on $Let$ \\
(3) $\G', \D \vdash e : \sigma$ \` by inversion on $Let$\\
(4) $\hasnolinearvars{\G'}$ \` by inversion on $Let$\\
(5) $\G, \G', \D \vdash e'[e/x] : \varphi$ \` by $\Delta$-var subst. lemma (2,3)\\
\end{tabbing}

\item[Case:] $\lletrec{\overline{x{:}_\Delta\sigma = e}}{e'} \longrightarrow e'\overline{[\lletrec{\overline{x{:}_\Delta\sigma = e}}{e}/x]}$
\begin{tabbing}
    (1) $\G, \ov{\G'}, \ov{\D} \vdash \lletrec{\overline{x{:}_\Delta\sigma =
    e}}{e'} : \vp$\\
    (2) $\G,\ov{\D}, \ov{\xD} \vdash e' : \vp$ \` by inversion on $Let$\\
    (3) $\ov{\G',\D, \ov{\xD} \vdash e : \s}$ \` by inversion on $Let$\\
    (4) $\ov{\hasnolinearvars{\G'}}$ \` by inversion on $Let$\\
    (5) TODO Magic\\
    (6) $\ov{\G', \D \vdash \lletrec{\ov{\xD = e}}{e}} : \s$\` TODO\\
    (7) $\G, \ov{\G'}, \ov{\D} \vdash e'\ov{[\lletrec{\ov{\xD = e}}{e}/x]} :
    \vp$ \` by $\D$-var subst. corollary\\
\end{tabbing}
% If \G, \ov{\D}, \ov{\xD} \vdash e' : \vp and \ov{\G', \D \vdash e : \s} then
% \G, \ov{\D}, \ov{\G'} \vdash e'\ov{[e/x]}

\item[Case:]
    $\ccase{K~\overline{e}}{z{:}_{\overline{\Delta}}\sigma}~\{\dots,K~\overline{x{:}_\pi\sigma}\}\ \longrightarrow e'\overline{[e/x]}[K~\overline{e}/z]$
\begin{tabbing}
(0) TODO.\\
(1) $\Gamma,\Gamma' \vdash \ccase{K~\overline{e}}{z{:}_{\overline{\Delta}}\sigma~\{\dots,K~\overline{x{:}_\pi\sigma}\to e'\}} : \varphi$ \` by assumption \\
(2) $\Gamma \vdash K~\overline{e} : \sigma$ \` by inversion on (case) \\
(3) $\Gamma', z{:}_\Delta\sigma \vdash_{alt} K~\overline{x{:}_\pi\sigma} \to e' : \sigma \Longrightarrow \varphi$ \` by inversion on (case) \\
% TODO: Como é que uso estes casos
(4) $K : \overline{\sigma \to_\pi} T~\overline{p} \in \Gamma'$ \` by inversion on (alt) \\
(5) $\Gamma',z{:}_\Delta\sigma,\overline{x{:}_\pi\sigma} \vdash e' : \varphi$ \` by inversion on (alt) \\
% TODO: Delta_i?
(6) $\Gamma, z{:}_{\Delta_i}\sigma, \Gamma' \vdash e'\overline{[e/x]} : \varphi$ \` by subst. lemma (2,5) \\
% TODO: HOW? Missing something (inversion on case). Hard!
(7) $\Gamma, \Gamma' \vdash e'\ov{[e/x]}[K~\ov{e}/z]$ \` by subst. lemma (6,2) \\
\end{tabbing}

\item[Case:] $e_1~e_2 \longrightarrow e_1'~e_2$
\begin{tabbing}
(1) $e_1 \longrightarrow e_1'$ \` by inversion on $\beta$-reduction \\
(2) $\Gamma,\Gamma' \vdash e_1~e_2 : \varphi$ \` by assumption \\
(3) $\Gamma \vdash e_1 : \sigma \to_\pi \varphi$ \` by inversion on ($\lambda E$) \\
(4) $\Gamma' \vdash e_2 : \sigma$ \` by inversion on ($\lambda E$) \\
(5) $\Gamma \vdash e_1' : \sigma \to_\pi \varphi$ \` by induction hypothesis in (3,1) \\
(6) $\Gamma, \Gamma' \vdash e_1'~e_2 : \varphi$ \` by ($\lambda E$) (4,3) \\
\end{tabbing}

\item[Case:] $e~\pi \longrightarrow e'~\pi$
\begin{tabbing}
(1) $e \longrightarrow e'$ \` by inversion on mult. $\beta$-reduction \\
(2) $\Gamma \vdash e~\pi : \sigma[\pi/p]$ \` by assumption \\
(3) $\Gamma \vdash e : \forall p.~\sigma$ \` by inversion on ($\Lambda E$) \\
(4) $\Gamma \vdash_{mult} \pi$ \` by inversion on ($\Lambda E$) \\
(5) $\Gamma \vdash e' : \forall p.~\sigma$ \` by induction hypothesis (3,1) \\
(6) $\Gamma \vdash e'~\pi : \sigma[\pi/p]$ \` by ($\Lambda E$) (5,4) \\
\end{tabbing}

\item[Case:] $\ccase{e}{z{:}_{\overline{\Delta}}\sigma~\{\rho_i\to e'_i\}} \longrightarrow \ccase{e'}{z{:}_{\overline{\Delta}}\sigma~\{\rho_i\to e'_i\}}$
\begin{tabbing}
(1) $e \longrightarrow e'$ \` by inversion on case reduction \\
(2) $\Gamma, \Gamma' \vdash \ccase{e}{z{:}_{\overline{\Delta}}\sigma~\{\rho_i\to e'_i\}} : \varphi$ \` by assumption \\
(3) $\Gamma \vdash e : \sigma$ \` by inversion on case \\
% TODO: HOW i?
(4) $\overline{\Gamma', z{:}_{\Delta_i}\sigma \vdash_{alt} \rho_i \to e'_i : \varphi}$ \` by inversion on case \\
(5) $\Gamma \vdash e' : \sigma$ \` by induction hypothesis (3,1) \\
(6) $\Gamma, \Gamma' \vdash \ccase{e'}{z{:}_{\overline{\Delta}}\sigma~\{\rho_i\to e'_i\}} : \varphi$ \` by case (4,3) \\
\end{tabbing}

\end{description}

\end{proof}

