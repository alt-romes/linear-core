\newcommand{\TypePreservationTheorem}{
\begin{theorem}[Type preservation]
\emph{If $\Gamma; \Delta \vdash e : \sigma$ and $e \longrightarrow e'$ then $\Gamma; \Delta \vdash e' : \sigma$}
\end{theorem}
}

\newcommand{\ProgressTheorem}{
\begin{theorem}[Progress]
\emph{Evaluation of a well-typed term does not block. If $\cdot; \cdot \vdash e :
  \sigma$ then $e$ is a value or there exists $e'$ such that $e \longrightarrow e'$.}
\end{theorem}
}

\newcommand{\WHNFConvSoundness}{
\begin{lemma}[Needs name]
Soundness of Not WHNF alternative typing wrt WHNF alternative typing\\
    If $\Gamma, \z[\irr{\D}]; \irr{\Delta}, \D' \vdash_{alt} \rho \to e :^z_{\irr{\D}} \sigma \Rrightarrow \vp$
    then $\Gamma, \z[\ov{\D_i}]; \ov{\Delta_i}, \D' \vdash_{alt} \rho \to e :^z_{\ov{\D_i}} \sigma \Mapsto \vp$ for \emph{any} $\ov{\D_i}$
\end{lemma}
}

\newcommand{\DeltaLinearRelationLemma}{
\begin{assumption}[Delta-Linear Something]
... If $\G,\xD; \D,\D' \vdash e : \s$ and $\D$ is consumed through $\xD$ in $e$
    then $\G; \D',\xl \vdash e :\s$.
\end{assumption}

\begin{assumption}[Reverse Delta-Linear Something]
... If $\G; \D',\xl \vdash e :\s$\\
    then $\G,\xD; \D,\D' \vdash e : \s$.
    and $\D$ is consumed through $\xD$ in $e$
\end{assumption}
}

\newcommand{\LinearSubstitutionLemma}{
\begin{lemma}[Substitution of linear variables preserves typing]
  If $\judg[\G][\D,\xl]{e}{\vp}$ and $\judg[\G][\D']{e}{\s}$
  then $\judg[\subst{\G}{\D'}{x}][\D,\D']{e[e'/x]}{\vp}$,
  where $\G[\D'/x]$ substitutes all occurrences of $x$ in the usage
  environments of variables in $\G$ by the linear variables in $\D'$.
  (really, $x$ couldn't appear anywhere else since $x$ is linear).
\end{lemma}
}

\newcommand{\UnrestrictedSubstitutionLemma}{
\begin{lemma}[Substitution of unrestricted variables preserves typing]
\emph{If $\Gamma, x{:}_\omega\sigma; \Delta \vdash e : \varphi$ and $\Gamma; \cdot \vdash e' : \sigma$ then $\G,\D \vdash e[e'/x] : \varphi$.}
\end{lemma}
}

\newcommand{\DeltaSubstitutionLemma}{
% usage environments only record linear variables.
\begin{lemma}[Substitution of $\Delta$-variables (with usage environments) preserves typing]
If \\ $\G,\xD; \D, \D' \vdash e : \varphi$ and $\G; \D \vdash e' : \sigma$ then $\G; \D, \D' \vdash e[e'/x] : \varphi$
% TODO: Write a paragraph about why this lemma has to have the same delta in the subst.
% Do we still need the $Delta = empty$ restriction?
% I think yes, and if we have $Delta = empty$ we never call delta substitution, but rather move the variable to be unrestricted and then call unr. subst. lemma
% Nevermind, I don't think there needs to be a distinction between unrestricted and empty-envs. Empty-envs can be simply trivially instantiated by Delta-var rule.
% And we needed it because the case-binder was handled differently, and its substitution when it was empty was ill-defined.
% That is no longer the case.
\end{lemma}
}


%%% OPTIMIZATIONS %%%

\newcommand{\BetaReductionTheorem}{
\begin{theorem}[$\beta$-reduction preserves types (and thus is sound)]~\\
If $\G; \D \vdash (\lambda \x[\pi][\s].~e)~e' : \vp$ then $\G; \D \vdash e[e'/x] : \vp$
\end{theorem}
}

\newcommand{\BetaReductionSharingTheorem}{
\begin{theorem}[Beta-reduction with sharing preserves types (and thus is sound)]~\\
    If $\G; \D \vdash (\lambda \xo.~e)~e' : \vp$ then $\G; \D \vdash \llet{x = e'}{e} : \vp$
    NB: We only apply this transformation when $x$ is unrestricted, otherwise beta-reduction without sharing is the optimization applied.
\end{theorem}
}

\newcommand{\BetaReductionMultTheorem}{
\begin{theorem}[Beta-reduction on multiplicity abstractions preserves types]~\\
    If $\G; \D \vdash (\Lambda p.~e)~\pi : \vp$ then $\G; \D \vdash e[\pi/p] : \vp$
\end{theorem}
}


\newcommand{\BinderSwapTheorem}{
\begin{theorem}[Binder-swap preserves types (and thus is sound)]~\\
If $\G \vdash \ccase{x}{\z[\ov{\D},x][\s]~\{\ov{\rho_i\to e_i}\}} : \vp$ then $\G \vdash \ccase{x}{\z[\ov{\D},x][\s]~\{\ov{\rho_i\to e_i[z/x]}\}} : \vp$
\end{theorem}
}

\newcommand{\InliningTheorem}{
\begin{theorem}[Inlining preserves types]~\\
If $\G; \D,\D' \vdash \llet{\xD = e}{e'} : \vp$ then $\G; \D, \D' \vdash \llet{\xD = e}{e'[e'/x]} : \vp$
\end{theorem}
}


\newcommand{\FloatInTheorem}{
\begin{theorem}[Float-in preserves types (and thus is sound)]~\\
Not trivial to define. Perhaps best done by case analysis
\end{theorem}
}

\newcommand{\FullLazinessTheorem}{
\begin{theorem}[Full-laziness preserves types (and thus is sound)]
    \emph{Not trivial to define. Perhaps best done by case analysis?}
\end{theorem}
}

\newcommand{\LocalTransformationsTheorem}{
\begin{theorem}[Local-transformations preserve types]
There are three lemmas for local transformations:
  \[
  \begin{array}{llcl}
  1. & \G \vdash (\llet{v = e}{b})~a : \vp & \Longrightarrow & \G \vdash \llet{v = e}{b~a} : \vp\\
  2. & \G \vdash \ccase{(\llet{v = e}{b})}{\dots} : \vp & \Longrightarrow & \G \vdash \llet{v = e}{\ccase{b}{\dots}} : \vp\\
  3. & \G \vdash \llet{x = (\llet{v = e}{b})}{c} : \vp & \Longrightarrow & \G \vdash \llet{v = e}{\llet{x = b}{c}} : \vp\\
  \end{array}
  \]
\end{theorem}
}

\newcommand{\CaseOfKnownConstructorTheorem}{
\begin{theorem}[Case-of-known-constructor preserves types (and thus is sound)]~\\
If $\G \vdash \ccase{K~\ov{e}}{\z[\ov{\D},\D_s][\s]~\{..., K~\ov{x} \to e_i\}} : \vp$ then $\G \vdash e_i\ov{[e/x]}[K~\ov{e}/z] : \vp$
\end{theorem}
}

\newcommand{\CaseOfCaseTheorem}{
\begin{theorem}[Case-of-case preserves types (and thus is sound)]~\\
\end{theorem}
}
