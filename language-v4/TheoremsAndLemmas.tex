\newcommand{\TypePreservationTheorem}{
\begin{theorem}[Type preservation]
If $\Gamma; \Delta \vdash e : \sigma$ and $e \longrightarrow e'$ then $\Gamma;
    \Delta \vdash e' : \sigma$
\end{theorem}
}

\newcommand{\ProgressTheorem}{
\begin{theorem}[Progress]
Evaluation of a well-typed term does not block.\\
If $\cdot; \cdot \vdash e : \sigma$ then $e$ is a value or there exists $e'$ such that $e \longrightarrow e'$.
\end{theorem}
}

\newcommand{\WHNFConvSoundness}{
\begin{lemma}[Irrelevance]
If $\Gamma, \z[\irr{\D}]; \irr{\Delta}, \D' \vdash_{alt} \rho \to e :^z_{\irr{\D}} \sigma \Rrightarrow \vp$\\
then $\Gamma, \z[\D^\dag]; \D^\dag, \D' \vdash_{alt} \rho \to e :^z_{\D^\dag} \sigma \Rightarrow \vp$, for \emph{any} $\D^\dag$
\end{lemma}
}

\newcommand{\DeltaLinearRelationLemma}{
\begin{assumption}[$\D \Rightarrow 1$]
A $\Delta$-variable can replace its usage environment $\D$ as a linear variable if $\D$ is decidedly consumed through it\\
If $\G,\xD; \D,\D' \vdash e : \s$ and $\D$ is consumed through $\xD$ in $e$
then $\G; \D',\xl \vdash e :\s$.
\end{assumption}
}

\newcommand{\LinearDeltaRelationLemma}{
\begin{assumption}[$1 \Rightarrow \D$]
    TODO These assumptions are wrong.
A linear variable can be moved to the unrestricted context as a $\D$-var with usage environment $\D$ by introducing $\D$ in the linear resources\\
If $\G; \D',\xl \vdash e :\s$
then $\G,\xD; \D,\D' \vdash e : \s$.
% and $\D$ is consumed through $\xD$ in $e$
\end{assumption}
}

\newcommand{\DeltaUnrestrictedRelationLemma}{
An unrestricted variable is equivalent to a $\D$-var with an empty usage environment.\\
\begin{assumption}[$\xo = x{:}_{\cdot}\s$]
$\xo = \x[\cdot]$ and $\G,\xo; \D \vdash e : \s \equiv \G,\x[\cdot]; \D \vdash e$
\end{assumption}
}


\newcommand{\LinearSubstitutionLemma}{
\begin{lemma}[Substitution of linear variables preserves typing]
  If $\judg[\G][\D,\xl]{e}{\vp}$ and $\judg[\G][\D']{e}{\s}$
  then $\judg[\subst{\G}{\D'}{x}][\D,\D']{e[e'/x]}{\vp}$
\end{lemma}
}

\newcommand{\LinearSubstitutionAltsLemma}{
\begin{sublemma}[Substitution of linear variables on case alternatives preserves typing]~\\
    If $\judg[\G][\D,\xl]{\rho\to e}{\s \Rightarrow \vp}[alt][\D_s][z]$ and
    $\judg[\G][\D']{e'}{\s}$ and
    $\D_s \subseteq \D,x$ \\then
    $\judg[\subst{\G}{\D'}{x}][\D,\D']{\rho \to e[e'/x]}{\s \Rightarrow \vp}[alt][\subst{\D_s}{\D'}{x}][z]$
\end{sublemma}
}

\newcommand{\UnrestrictedSubstitutionLemma}{
\begin{lemma}[Substitution of unrestricted variables preserves typing]
\emph{If $\Gamma, x{:}_\omega\sigma; \Delta \vdash e : \varphi$ and $\Gamma; \cdot \vdash e' : \sigma$ then $\G,\D \vdash e[e'/x] : \varphi$.}
\end{lemma}
}

\newcommand{\UnrestrictedSubstitutionAltsLemma}{
\begin{sublemma}[Substitution of unrestricted variables on case alternatives
    preserves typing]
If $\G, \xo; \D \vdash_{alt} \rho \to e :^z_{\Delta_s} \s' \Rightarrow \vp$ and $\G; \D \vdash e' : \s$ and
    then $\G; \D \vdash_{alt} \rho \to e[e'/x] :^z_{\Delta_s} \s' \Rightarrow \vp$
\end{sublemma}
}

\newcommand{\DeltaSubstitutionLemma}{
% usage environments only record linear variables.
\begin{lemma}[Substitution of $\Delta$-variables preserves typing]
If $\G,\xD; \D, \D' \vdash e : \varphi$ and $\G; \D \vdash e' : \sigma$ then $\G; \D, \D' \vdash e[e'/x] : \varphi$
% TODO: Write a paragraph about why this lemma has to have the same delta in the subst.
% Do we still need the $Delta = empty$ restriction?
% I think yes, and if we have $Delta = empty$ we never call delta substitution, but rather move the variable to be unrestricted and then call unr. subst. lemma
% Nevermind, I don't think there needs to be a distinction between unrestricted and empty-envs. Empty-envs can be simply trivially instantiated by Delta-var rule.
% And we needed it because the case-binder was handled differently, and its substitution when it was empty was ill-defined.
% That is no longer the case.
\end{lemma}
}

\newcommand{\DeltaSubstitutionAltsLemma}{
\begin{sublemma}[Substitution of $\Delta$-bound variables on case alternatives
    preserves typing]
If $\G,\xD; \D,\D' \vdash_{alt} \rho \to e :^z_{\Delta_s} \s' \Rightarrow \vp$ and
    $\G; \D \vdash e' : \s$ and $\Delta_s \subseteq (\Delta,\Delta')$ then $\G; \D, \D' \vdash_{alt} \rho \to e[e'/x] :^z_{\Delta_s} \s' \Rightarrow \vp$
\end{sublemma}
}

%%% OPTIMIZATIONS %%%

\newcommand{\BetaReductionTheorem}{
\begin{theorem}[$\beta$-reduction preserves types]~\\
If $\G; \D \vdash (\lambda \x[\pi][\s].~e)~e' : \vp$ then $\G; \D \vdash e[e'/x] : \vp$
\end{theorem}
}

\newcommand{\BetaReductionSharingTheorem}{
\begin{theorem}[$\beta$-reduction with sharing preserves types]~\\
    If $\G; \D \vdash (\lambda \xo.~e)~e' : \vp$ then $\G; \D \vdash \llet{x = e'}{e} : \vp$
    NB: We only apply this transformation when $x$ is unrestricted, otherwise beta-reduction without sharing is the optimization applied.
\end{theorem}
}

\newcommand{\BetaReductionMultTheorem}{
\begin{theorem}[$\beta$-reduction on multiplicity abstractions preserves types]~\\
    If $\G; \D \vdash (\Lambda p.~e)~\pi : \vp$ then $\G; \D \vdash e[\pi/p] : \vp$
\end{theorem}
}


\newcommand{\BinderSwapTheorem}{
\begin{theorem}[Binder-swap preserves types]~\\
If $\G; \D \vdash \ccase{x}{z~\{\ov{\rho_i\to e_i}\}} : \vp$ then $\G; \D \vdash \ccase{x}{z~\{\ov{\rho_i\to e_i[z/x]}\}} : \vp$
\end{theorem}
}

\newcommand{\ReverseBinderSwapTheorem}{
\begin{theorem}[Reverse-binder-swap preserves types]~\\
If $\G; \D \vdash \ccase{x}{z~\{\ov{\rho_i\to e_i}\}} : \vp$ then $\G; \D \vdash \ccase{x}{z~\{\ov{\rho_i\to e_i[x/z]}\}} : \vp$
\end{theorem}
}

\newcommand{\InliningTheorem}{
\begin{theorem}[Inlining preserves types]~\\
If $\G; \D,\D' \vdash \llet{\xD = e}{e'} : \vp$ then $\G; \D, \D' \vdash \llet{\xD = e}{e'[e/x]} : \vp$
\end{theorem}
}


\newcommand{\FloatInTheorem}{
\begin{theorem}[Float-in preserves types]~\\
HELP: How to define?
\end{theorem}
}

\newcommand{\FullLazinessTheorem}{
\begin{theorem}[Full-laziness preserves types]~\\
    If $\G; \D,\D' \vdash \lambda \y[\pi].~\llet{\xD = e}{e'} : \vp$ and $y$ does not occur in $e$ \\then $\G;\D,\D' \vdash \llet{\xD=e}{\lambda \y[\pi].~e'}$
\end{theorem}
}

\newcommand{\LocalTransformationsTheorem}{
\begin{theorem}[Local-transformations preserve types]
There are three lemmas for local transformations:
  \[
  \begin{array}{llcl}
  1. & \G; \D \vdash (\llet{v = e}{b})~a : \vp & \Rightarrow & \G; \D \vdash \llet{v = e}{b~a} : \vp\\
  2. & \G; \D \vdash \ccase{(\llet{v = e}{b})}{\dots} : \vp & \Rightarrow & \G; \D \vdash \llet{v = e}{\ccase{b}{\dots}} : \vp\\
  3. & \G; \D \vdash \llet{x = (\llet{v = e}{b})}{c} : \vp & \Rightarrow & \G; \D \vdash \llet{v = e}{\llet{x = b}{c}} : \vp\\
  \end{array}
  \]
\end{theorem}
}

\newcommand{\CaseOfKnownConstructorTheorem}{
\begin{theorem}[Case-of-known-constructor preserves types]~\\
If $\G; \D, \D' \vdash \ccase{K~\ov{e}}{\z[\D][\s]~\{..., K~\ov{x} \to e_i\}} : \vp$ then $\G; \D,\D' \vdash e_i\ov{[e/x]}[K~\ov{e}/z] : \vp$
\end{theorem}
}

\newcommand{\CaseOfCaseTheorem}{
\begin{theorem}[Case-of-case preserves types]~\\
    If $\G; \D, \D',\D'' \vdash \ccase{\ccase{e_c}{\z[\D]~\{\ov{\rho_{c_i} \to e_{c_i}}\}}}{\var[w][\irr{\D,\D'}][\s']~\{\ov{\rho_i \to e_i}\}} : \vp$\\
    then $\G; \D, \D',\D'' \vdash \ccase{e_c}{\z[\D]~\{\ov{\rho_{c_i} \to \ccase{e_{c_i}}{\var[w][\irr{\D,\D'}][\s']~\{\ov{\rho_i \to e_i}\}}}\}} : \vp$\\
\end{theorem}
}

\newcommand{\EtaExpansionTheorem}{
\begin{theorem}[$\eta$-expansion preserves types]~\\
    If $\G; \D \vdash f : \s \to_\pi \vp$ then $\G; \D \vdash \lambda x.~f~x : \s \to_\pi \vp$
\end{theorem}
}

\newcommand{\EtaReductionTheorem}{
\begin{theorem}[$\eta$-reduction preserves types]~\\
    If $\G; \D \vdash \lambda x.~f~x : \s \to_\pi \vp$ then $\G; \D \vdash f : \s \to_\pi \vp$
\end{theorem}
}
