\newcommand{\SyntaxTypes}{
\ensuremath{
\begin{array}{lcll}
    \ROUNDTWO{\tau,\sigma}  & ::=  & T~\overline{p}         & \textrm{Datatype} \\
                    & \mid & \ROUNDTWO{\tau} \to_\pi \sigma & \textrm{Function with multiplicity}\\
                    & \mid & \forall p.~\ROUNDTWO{\tau}     & \textrm{Multiplicity universal scheme}\\
                   % TODO: Eventually Coercions
\end{array}
}
}

\newcommand{\SyntaxTerms}{
\ensuremath{
\begin{array}{lcll}                                             
    e                & ::=  & x,y,z \mid K                                                                 & \textrm{Variables and data constructors}\\
                     & \mid & \Lambda p.~e~\mid~e~\pi                                          & \textrm{Multiplicity abstraction/application}\\
                     & \mid & \lambda x{:}_\pi\ROUNDTWO{\tau}.~e~\mid~e~e'                           & \textrm{Term abstraction/application}\\
                     & \mid & \llet{x{:}_{\Delta}\sigma = e}{e'}
                                                                                                           &
                                                                                                             \textrm{Let
                                                                                                             \ROUNDTWO{with
                                                                                                             usage
                                                                                                             environment}}
  \\
                     & \mid & \lletrec{\overline{x{:}_{\Delta}\sigma =
                              e}}{e'}             & \textrm{Recursive
                                                    Let \ROUNDTWO{with
                                                                                                             usage
                                                                                                             environment}} \\
                     & \mid &
                     \ccase{e}{z{:}_{\Delta}\ROUNDTWO{\tau}~\{\overline{\rho\ROUNDTWO{\Rightarrow} e'}\}}   & \textrm{Case} \\
%    p               & ::=  & K~\overline{b{:}\kappa}~\overline{x{:}\sigma}                    & \textrm{Pattern} \\
    \rho             & ::=  & K~\overline{x{:}_\pi\ROUNDTWO{\tau}} \mid \_                              & \textrm{Pattern and wildcard} \\
% Currently we don't care about the existential multiplicity variables, but later on we might
\end{array}
}
}

\newcommand{\SyntaxEnvironments}{
\ensuremath{
\begin{array}{lcll}
% TODO: Introduzir tagged resources
  \Gamma   & ::=  & \ROUNDTWO{\cdot \mid \Gamma,x{:}_\omega\tau \mid
                    \Gamma,K{:}\tau \mid \Gamma,p \mid
                    \Gamma,z{:}_{\Delta}\tau} & \ROUNDTWO{\textrm{Unrestricted
                                                typing environment}} \\
  \Delta   & ::=  & \cdot \mid \Delta,x{:}_\pi\tau \mid
                    \Delta,[x{:}_\pi\tau] & \ROUNDTWO{\textrm{Linear
                                            typing environment}} \\
  % \delta   & ::=  & \cdot \mid \delta,z{:}_{\Delta}\sigma & \textrm{$\Delta$-bound variables}\\
         % & \mid & \Gamma,x{:}_{\overline{\Delta}}\sigma & \textrm{Case bound variables} -- NOPE
         % &  \mid & \Gamma,K{:}\sigma      &         & \textrm{Data constructor}\\
         % &  \mid & \Gamma,p               &         & \textrm{Multiplicity variable}\\
\end{array}
}
}

\newcommand{\SyntaxFull}{
\begin{figure}[ht]
\begin{framed}
\[
{\small
  \begin{array}{l}
%
\textbf{Types} \\
\SyntaxTypes\\
%
\textbf{Terms}\\
\SyntaxTerms\\\\
%
\textbf{Environments}\\
\SyntaxEnvironments\\\\
%
\textbf{Multiplicities} \qquad\qquad \textbf{Declarations}\\
\begin{array}{lcl}
  \pi & ::= & 1 \mid \omega \mid p \qquad decl  ::=  \datatype{T~\overline{p}}{\overline{K:\overline{\ROUNDTWO{\tau} \to_\pi}~T~\overline{p}}} %\mid \pi + \mu \mid \pi \cdot \mu\\
% We don't use + and \cdot yet, but we will
\end{array}
%
% \textbf{Usage Environments}\\
% % something like dual numbers, in that we now have the two components of each usage environment.
% \begin{array}{lcl}
%   \Delta & ::= \cdot\mid\Delta_1+\Delta_2\mid\pi\Delta\\
% \end{array}\\\\
%
% \textbf{Declarations}\\
% \begin{array}{lcl}
%  % pgm & ::= & \overline{decl}; e \\
%   decl & ::= & \datatype{T~\overline{p}}{\overline{K:\overline{\sigma \to_\pi}~T~\overline{p}}}
% \end{array}
%
\end{array}}
\]
               
\end{framed}
\caption{Linear Core Syntax}
\label{fig:full-linear-core-syntax}
\end{figure}
}

