
\begin{figure}[h]
\begin{framed}
\[
\begin{array}{l}
%
\textbf{Types} \\
\begin{array}{lcll}
    \varphi,\sigma  & ::=  & T~\overline{p}         & \textrm{Datatype} \\
                    & \mid & \varphi \to_\pi \sigma & \textrm{Function with multiplicity}\\
                    & \mid & \forall p.~\varphi     & \textrm{Multiplicity universal scheme}\\
                   % TODO: Eventually Coercions
\end{array}\\\\
%
\textbf{Terms}\\
\begin{array}{lcll}
    u                & ::=  & x,y,z \mid K                                                     & \textrm{Variables and data constructors}\\
    e                & ::=  & u                                                                & \textrm{Term atoms}\\
                     & \mid & \Lambda p.~e~\mid~e~\pi                                          & \textrm{Multiplicity abstraction/application}\\
                     & \mid & \lambda x{:}_\pi\sigma.~e~\mid~e~e'                           & \textrm{Term abstraction/application}\\
                     & \mid & \llet{x{:}_{\Delta}\sigma = e}{e'}                              & \textrm{Let} \\
                     & \mid & \lletrec{\overline{x{:}_{\Delta}\sigma = e}}{e'}             & \textrm{Recursive Let} \\
                     & \mid &
                     \ccase{e}{z{:}_{\Delta}\sigma~\{\overline{\rho\to e'}\}}   & \textrm{Case} \\
                     &      &                                                                  & \\
%    p               & ::=  & K~\overline{b{:}\kappa}~\overline{x{:}\sigma}                    & \textrm{Pattern} \\
    \rho             & ::=  & K~\overline{x{:}_\pi\sigma} \mid \_                              & \textrm{Pattern and wildcard} \\
% Currently we don't care about the existential multiplicity variables, but later on we might
\end{array}\\\\
%
\textbf{Environments}\\
\begin{array}{lcll}
% TODO: Introduzir tagged resources
  \Gamma   & ::=  & \cdot \mid \Gamma,x{:}_\omega\sigma \mid \Gamma,K{:}\sigma \mid \Gamma,p \mid z{:}_{\Delta}\sigma & \textrm{Unrestricted (delta-)variables} \\
  \Delta   & ::=  & \cdot \mid \Delta,x{:}_\pi\sigma \mid \Delta,[x{:}_\pi\sigma] & \textrm{Linear (and irrelevant) resources} \\
  % \delta   & ::=  & \cdot \mid \delta,z{:}_{\Delta}\sigma & \textrm{$\Delta$-bound variables}\\
         % & \mid & \Gamma,x{:}_{\overline{\Delta}}\sigma & \textrm{Case bound variables} -- NOPE
         % &  \mid & \Gamma,K{:}\sigma      &         & \textrm{Data constructor}\\
         % &  \mid & \Gamma,p               &         & \textrm{Multiplicity variable}\\
\end{array}\\\\
%
\textbf{Multiplicities}\\
\begin{array}{lcll}
  \pi, \mu & ::= & 1 \mid \omega \mid p \mid \pi + \mu \mid \pi \cdot \mu\\
% We don't use + and cdot yet, but we will
\end{array}\\\\
%
\textbf{Usage Environments}\\
% something like dual njumbers, in that we now have the two components of each usage environment.
\begin{array}{lcl}
  \Delta & ::= \cdot\mid\Delta_1+\Delta_2\mid\pi\Delta\\
\end{array}\\\\
%
\textbf{Declarations}\\
\begin{array}{lcl}
  pgm & ::= & \overline{decl}; e \\
  decl & ::= & \datatype{T~\overline{p}}{\overline{K:\overline{\sigma \to_\pi}~T~\overline{p}}}
\end{array}
%
\end{array}
\]
\end{framed}
\caption{Linear Core* Syntax}
\label{linear-core-syntax}
\end{figure}

