% Rules that have variables in the context aren't considered!

\progressthm*

\begin{proof}
By structural induction on the (only) typing derivation

\begin{description}

\item[Case:] $\Lambda I$
\begin{tabbing}
(1) $\cdot; \cdot \vdash (\Lambda p.~e) : \forall p.~\sigma$ \` by assumption \\
(2) $(\Lambda p.~e)~\textrm{is a value}$ \` by definition \\
\end{tabbing}

\item[Case:] $\Lambda E$
\begin{tabbing}
(1) $\cdot; \cdot \vdash e_1~\pi : \sigma[\pi/p]$ \` by assumption \\
(2) $\cdot; \cdot \vdash e_1 : \forall p.~\sigma$ \` by inversion on ($\Lambda E$) \\
(3) $\cdot; \cdot \vdash_{mult} \pi$ \` by inversion on ($\Lambda E$) \\
(4) $e_1~\textrm{is a value or}~\exists e_1'.~e_1 \longrightarrow e_1'$ \` by the induction hypothesis (2) \\
\textrm{Subcase $e_1$ is a value:}\\
(5) $e_1 = \Lambda p.~e_2$ \` by the canonical forms lemma (2) \\
(6) $(\Lambda p.~e_2)~\pi \longrightarrow e_2[\pi/p]$ \` by $\beta$-reduction on multiplicity (5,3) \\
\textrm{Subcase $e_1 \longrightarrow e_1'$:}\\
% TODO: Have I've concluded two different things in the proof because
% the reductions don't match (on one we have explicit substitution)?
(5) $e_1~\pi \longrightarrow e_1'~\pi$ \` by context reduction on mult. application \\
\end{tabbing}

\item[Case:] $\lambda I$
\begin{tabbing}
(1) $\cdot; \cdot \vdash (\lambda x{:}_\pi\sigma.~e) : \sigma\to_\pi\varphi$ \` by assumption \\
(2) $(\lambda x{:}_\pi\sigma.~e)~\textrm{is a value}$ \` by definition \\
\end{tabbing}

\item[Case:] $\lambda E$
\begin{tabbing}
(1) $\cdot; \cdot \vdash e_1~e_2 : \varphi$ \` by assumption \\
(2) $\cdot; \cdot \vdash e_1 : \sigma \to_\pi \varphi$ \` by inversion on ($\lambda E$) \\
(3) $\cdot; \cdot \vdash e_2 : \sigma$ \` by inversion on ($\lambda E$) \\
(4) $e_1~\textrm{is a value or}~\exists e_1'.~e_1\longrightarrow e_1'$ \` by the induction hypothesis (2) \\
\textrm{Subcase $e_1$ is a value:}\\
(5) $e_1 = \lambda x{:}_\pi\sigma.~e$ \` by the canonical forms lemma \\
(6) $e_1~e_2 \longrightarrow e[e_2/x]$ \` by term $\beta$-reduction (5,3) \\
\textrm{Subcase $e_1\longrightarrow e_1'$:}\\
(5) $e_1~e_2\longrightarrow e_1'~e_2$ \` by context reduction on term application \\
\end{tabbing}

\item[Case:] $Let$
\begin{tabbing}
(1) $\cdot \vdash \llet{x{:}_\Delta\sigma = e}{e'} : \varphi$ \` by assumption \\
(2) $\llet{x{:}_\Delta\sigma = e}{e'}\longrightarrow e'[e/x]$ \` by definition\\
\end{tabbing}

\item[Case:] $LetRec$
\begin{tabbing}
(1) $\cdot; \cdot \vdash \lletrec{\overline{x_i{:}_\Delta\sigma_i = e_i}}{e'} : \varphi$ \` by assumption \\
(2) $\lletrec{\overline{x_i{:}_\Delta\sigma_i = e_i}}{e'} \longrightarrow
    e'\overline{[\lletrec{\overline{x_i{:}_\Delta\sigma_i = e_i}}{e_i}/x]}$ \` by definition\\
\end{tabbing}

\item[Case:] $CaseWHNF$ and $CaseNotWHNF$
\begin{tabbing}
    (1) $\cdot; \cdot \vdash \ccase{e}{z{:}_{\cdot}\sigma~\{\overline{\rho_i \to e_i}\}} : \varphi$ \` by assumption \\
    (2) $\cdot; \cdot \vdash e : \s$ \` by inversion of $CaseWHNF$ or $CaseNotWHNF$\\
    (4) $\overline{\cdot, z{:}_{\cdot}; \cdot \sigma \vdash_{alt} \rho_i\to e_i :^z_\cdot \sigma \Rightarrow \varphi}$ \` by inversion of $CaseWHNF$ or $CaseNotWHNF$\\
    (5) $e~\textrm{is a value or}~\exists e'.~e \longrightarrow e'$ \` by induction hypothesis (2) \\
    Subcase $e$ is a value\\
    % TODO: this should rather be whether $e$ is in WHNF,\\
    % and there should be a better connection between values and WHNF explicit.\\
    (6) $e_1 = K~\overline{e}$ \` by canonical forms lemma \\
    (7) $e$ is in WHNF\` by (6)\\ % (TODO: match correctly value and WHNF)\\
    (8) $\overline{\rho_i \to e_i}$ is a complete pattern \` by coverage checker\\
    (9) $\ccase{K~\ov{e}}{z{:}_{\cdot}\sigma~\{\overline{\rho_i \to e_i}\}} \longrightarrow e_i\overline{[e/x]}[K~\overline{e}/z]$\\\` by case reduction on pattern or wildcard\\
    Subcase $\exists e'.~e \longrightarrow e'$\\
    (6) $e$ is definitely not in WHNF\\
    (7) $\ccase{e}{z{:}_{\cdot}\sigma~\{\overline{\rho_i \to e_1}\}} \longrightarrow \ccase{e'}{z{:}_{\cdot}\sigma~\{\rho_i \to e_i\}}$ \` by ctx. case reduction\\
\end{tabbing}

\end{description}

\end{proof}

