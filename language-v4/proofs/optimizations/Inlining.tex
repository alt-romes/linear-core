Inlining substitutes occurrences of a let-bound $\D$-variable $x$
with the expression $e$ it is bound to, which determines its usage environment
$\D$. Intuitively, in the let body $e'$, $x$ can occur once or not at all: if
$x$ occurs, then the linear resources $\D$ used indirectly through $x$ are used
via $e$ instead; if $x$ does not occur, then the resources $\D$ are already
used linearly in $e'$ and the substitution is a no-op.

\InliningTheorem

\begin{proof}~

\begin{tabbing}
    (1) $\G;\D,\D' \vdash \llet{\xD = e}{e'} : \vp$\\
    (2) $\G,\D \vdash e : \s$\` by inv. on (let)\\
    (3) $\G,\xD; \D, \D' \vdash e' : \vp$\` by inv. on (let)\\
    (4) $\G;\D,\D' \vdash e'[e/x] : \vp$\` by $\D$-subst. lemma (2,3)\\
    (5) $\G,\xD; \D,\D' \vdash e'[e/x] : \vp$ \` by (admissible) $Weaken_\Delta$\\
    (6) $\G;\D,\D' \vdash \llet{\xD = e}{e'[e/x]} : \vp$\` by (let) (2,5)\\
\end{tabbing}
\end{proof}

