The $\eta$-conversion transformations are $\eta$-expansion and
$\eta$-reduction. In both transformations, linearity is preserved since the
resources used to type the function $f$ do not change neither when the lambda
and its argument are removed, nor when we add a lambda and apply $f$ to the
bound argument.

\EtaExpansionTheorem

\begin{proof}~
\begin{tabbing}
    Subcase $f$ is linear\\
    (1) $\G; \D \vdash f : \s \lolli \vp$\\
    (2) $\G; \xl \vdash x : \s$\\
    (3) $\G; \D, \xl \vdash f~x : \vp$\`by $\lambda E$\\
    (4) $\G; \D \vdash (\lambda \xl.~f~x) : \s \lolli \vp$\`by $\lambda I$\\
    Subcase $f$ is unrestricted\\
    As above but $x$ is introduced in $\G$ and functions are unrestricted
\end{tabbing}
\end{proof}

\EtaReductionTheorem

\begin{proof}~
\begin{tabbing}
    (1) $\G; \D \vdash (\lambda \x[\pi].~f~x) : \s \to_\pi \vp$\\
    Subcase $\pi = 1$\\
    (2) $\G; \D, \xl \vdash f~x : \vp$\`by inv. on $\lambda I$\\
    (3) $\G; \D \vdash f : \s \to_1 \vp$\`by inv. on $\lambda E$\\
    Subcase $\pi = \omega$\\
    As above but $x$ is introduced in $\G$
\end{tabbing}
\end{proof}

