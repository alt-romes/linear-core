$\beta$-reduction evaluates a $\lambda$-abstraction application by substituting
the $\lambda$-bound variable $x$ with the argument $e'$ in the body of the
$\lambda$-abstraction $e$.
%
We consider two definitions of $\beta$-reductions, one that substitutes all
occurrences of a variable by an expression, as in call-by-name, and other which
creates a lazy let binding to share the result of computing the argument
expression amongst uses of the variable, as in call-by-need.

The first kind of $\beta$-reduction on term abstractions can be seen to
preserve linearity by case analysis. When the function is linear, the binding
$x$ is used exactly once in the body of the lambda, thus can be substituted by
an expression typed with linear resources, since the expression is guaranteed
to be used exactly once in place of $x$. The proof is direct by type preservation.

\BetaReductionTheorem

\begin{proof}~

\begin{tabbing}
% Using preservation
    (1) $\G; \D \vdash (\lambda \x[\pi].~e)~e' : \vp$\\
    (2) $(\lambda \x[\pi].~e) e' \longrightarrow e[e'/x]$\\
    (3) $\G; \D \vdash e[e'/x] : \vp$ \` by type preservation theorem (1,2)\\

% Manually
%    Subcase $\pi = 1,p$\\
%    (1) $\G,\G' \vdash (\lambda \x[1,p][\s].~e)~e' : \vp$\\
%    (2) $\G \vdash (\lambda \x[1,p][\s].~e) : \s \to_{1,p} \vp$\` by inv. on $\lambda E_{1,p}$\\
%    (3) $\G' \vdash e' : \s$ \` by inv. on $\lambda E_{1,p}$\\
%    (4) $\G,\x[1,p][\s] \vdash e : \vp $\`by inv. on $\lambda I$\\
%    (5) $\x[1,p][\s] \notin \G$\`by inv. on $\lambda I$\\
%    (6) $\G,\G' \vdash e[e'/x] : \vp$\` by lin. subst. lemma (4,3) and (5)\\
%    Subcase $\pi = \omega$\\
%    (1) $\G,\G' \vdash (\lambda \xo.~e)~e' : \vp$\\
%    (2) $\G \vdash (\lambda \xo.~e) : \s \to_\omega \vp$\` by inv. on $\lambda E_\omega$\\
%    (3) $\G' \vdash e' : \s$ \` by inv. on $\lambda E_\omega$\\
%    (4) $\hasnolinearvars{\G'}$ \` by inv. on $\lambda E_\omega$\\
%    (5) $\G,\xo \vdash e : \vp$\` by inv. on $\lambda I$\\
%    (6) $\xo \notin \G$\` by inv. on $\lambda I$\\
%    (7) $\G,\G' \vdash e[e'/x] : \vp$\` by unr. subst. lemma (5,3,4)\\
\end{tabbing}
\end{proof}

\noindent We assume the $\beta$-reduction with sharing (i.e. the one that creates a let
binding) is only applicable when the $\lambda$-abstraction has an unrestricted
function type. Otherwise, the call-by-name $\beta$-reduction is always
favourable, as we know the resource to be used exactly once and hence sharing
would be counterproductive, and result in an unnecessary heap allocation.
%
Consequently, the argument to the function must be unrestricted (hence use no
linear resources) for the term to be well-typed, and so it vacuously follows
that linearity is preserved by this transformation.

\BetaReductionSharingTheorem

\begin{proof}~

\begin{tabbing}
    (1) $\G;\D \vdash (\lambda \xo.~e)~e' : \vp$\\
    (2) $\G; \D \vdash (\lambda \xo.~e) : \s \to_\omega \vp$\` by inv. on $\lambda E_\omega$\\
    (3) $\G; \cdot \vdash e' : \s$ \` by inv. on $\lambda E_\omega$\\
    (4) $\G,\xo; \D \vdash e : \vp$\` by inv. on $\lambda I$\\
    (5) $\G, \x[\cdot][\s]; \D \vdash e : \vp$\` by Lemma~\ref{lem:undelta}\\
    (6) $\G,\G' \vdash \llet{x = e'}{e} : \vp$\` by let (5,3)\\
\end{tabbing}
\end{proof}

\noindent Finally, $\beta$-reduction on multiplicity abstractions is also type
preserving. The argument of the application is a multiplicity rather than an
expression, so no resources are needed to type it, and, since the body of the
lambda must treat the multiplicity $p$ as though it were linear, the body uses
the argument linearly regardless of the instantiation of $p$ at $\pi$.

\BetaReductionMultTheorem

\begin{proof}
    Trivial by invoking preservation using the $\Lambda$ application reduction and the assumption
\end{proof}


