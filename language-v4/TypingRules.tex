% We don't need explicit weaking and contract: contexts without context are always split, contexts without weaking must be empty in the var rule.
\newcommand{\TypeVarOmega}{
    \infer*[right=($Var_\omega$)]
    {\,}
    {\judg[\G,\xo][\cdot]{x}{\s}}
}

\newcommand{\TypeLinearVar}{
    \infer*[right=($Var_1$)]
    {\,}
    {\judg[\G][\xl]{x}{\s}}
}

\newcommand{\TypeVarDelta}{
    \infer*[right=($Var_{\D}$)]
    {\,}
    {\judg[\G,\xD][\D]{x}{\s}}
}

% TODO: split tagged stuff and proof irrelevant stuff
\newcommand{\TypeVarSplit}{
    \infer*[right=($Split$)]
    {\judg[\G][\D,\x]{e}{\vp} \and \hasargs{K}{n}}
    {\judg[\G][\D,\ov{\xt{K_i}}^n]{x}{\s}}
}

\newcommand{\TypeMultLamIntro}{
    \infer*[right=($\Lambda I$)]
    {\judg[\G,p]{e}{\s} \and p\notin\Gamma}
    {\judg{\Lambda p.~e}{\forall p.~\s}}
}

\newcommand{\TypeMultLamElim}{
    \infer*[right=($\Lambda E$)]
    {\judg{e}{\forall p.~\s} \and \Gamma\vdash_{mult}\pi}
    {\judg{e~\pi}{\s[\pi/p]}}
}

\newcommand{\TypeLamIntroL}{
    \infer*[right=($\lambda I_1$)]
    {\judg[\G][\D,\xl]{e}{\vp} \and x\notin\Delta}
    {\judg{\lambda \xl.~e}{\s\to_{1}\vp}}
}

\newcommand{\TypeLamIntroW}{
    \infer*[right=($\lambda I_\omega$)]
    {\judg[\G,\xo]{e}{\vp} \and x\notin\Delta}
    {\judg{\lambda \xo.~e}{\s\to_{\omega}\vp}}
}

\newcommand{\TypeLamElimL}{
    \infer*[right=($\lambda E_1$)]
    {\judg[\G][\D]{e}{\s\to_{1}\vp} \and \judg[\G][\D']{e'}{\s}}
    {\splitjudg{e~e'}{\vp}}
}

\newcommand{\TypeLamElimW}{
% TODO: It's not entirely obvious whether we should split \delta and give some to the RHS
% Any \delta occurring in e' must have an empty usage environment and can be
% used unrestrictedly, so it could perhaps make more sense to move the variable from \delta to \Gamma
    \infer*[right=($\lambda E_\omega$)]
    {\judg{e}{\s\to_{\omega}\vp} \and \judg[\G][\cdot]{e'}{\s}}
    {\judg{e~e'}{\vp}}
}

\newcommand{\TypeLet}{
    % \infer*[right=($Let$)]
    \infer[(Let)]
    {\judg[\G][\D]{e}{\s} \\ \judg[\G,\xr{\D}{\s}][\D,\D']{e'}{\vp}}
    {\judg[\G][\D,\D']{\llet{\x[\D] = e}{e'}}{\vp}}
}


\newcommand{\TypeLetRec}{
    % \infer*[right=($LetRec$)]
  \infer[(LetRec)]
    {\ov{\judg[\G,\ov{\xr{\D}{\s}}][\D]{e}{\s}} \\ \judg[\G,\ov{\xr{\D}{\s}}][\D,\D']{e'}{\vp}}
    {\judg[\G][\D,\D']{\lletrec{\ov{\x[\D] = e}}{e'}}{\vp}}
}

\newcommand{\TypeCaseWHNF}{
    \mprset{flushleft}
    \infer[(CaseWHNF)]
    {\textrm{e is in \emph{WHNF}} \\ \judg{e}{\s}
    \\ \ov{\judg[\G,\zD][\D,\D']{\rho\to e'}{\s \Longrightarrow \vp}[alt][\D][z]}}
    {\judg[\G][\D,\D']{\ccase{e}{\zD~\{\ov{\rho \to e'}\}}}{\vp}}
}

\newcommand{\TypeCaseNotWHNFIncorrect}{
    \mprset{flushleft}
    \infer[(CaseNotWHNF)]
    {\textrm{e is definitely not in \emph{WHNF}} \\
    \judg{e}{\s}
    \\ \ov{\judg[\subst{\G}{\irr{\D}}{\D},z{:}_{\irr{\D}}\s][\irr{\D},\D']{\rho\to e'}{\s \Longrightarrow \vp}[alt][\irr{\D}][z]}}
    {\splitjudg{\ccase{e}{\z[\irr{\D}]~\{\ov{\rho \to e'}\}}}{\vp}}
    % TODO: Add that [d1] entails moving all delta usages to [delta] usages in all vars i d1
}

\newcommand{\TypeCaseNotWHNF}{
    \mprset{flushleft}
    \infer[(CaseNotWHNF)]
    {\textrm{e is definitely not in \emph{WHNF}} \\
    \judg{e}{\s}
    \\ \ov{\judg[\G,z{:}_{\irr{\D}}\s][\irr{\D},\D']{\rho\to e'}{\s \Longrightarrow \vp}[alt][\irr{\D}][z]}}
    {\splitjudg{\ccase{e}{\z[\irr{\D}]~\{\ov{\rho \to e'}\}}}{\vp}}
    % TODO: Add that [d1] entails moving all delta usages to [delta] usages in all vars i d1
}

% We no longer do this, because with beta-reduction this isn't sound.
% Great, we become uniform in that variables are considered not in WHNF
\newcommand{\TypeCaseVar}{
    \mprset{flushleft}
    \infer[(CaseVar)]
    {\textrm{x is a \emph{var}} \\ \judg{x}{\s}
    \\ \ov{\judg[\G][\D',\xl][\d',z{:}_x\s]{\rho\to e'}{\s \Longrightarrow \vp}[alt][x][z]}}
    {\splitjudg{\ccase{x}{\z[x]~\{\ov{\rho \to e'}\}}}{\vp}}
    % TODO: Explain that \delta becomes irrelevant since all its variables were
    % consumed. Perhaps there's a case to be made about pseudo-unrestricted
    % delta vars being moved here.
}

% TODO: For the next three rules, do I really need the K \in \Gamma bit? At least
% in the preservation to invoke constructor application lemma I do.

\newcommand{\TypeAltNIncorrect}{
    \mprset{flushleft}
    \infer[(AltN)]
    { \judg[\G,\ov{\xo},\ov{y_i{:}_{\lctag{\D_s}{K_i}}\s_i}^n][\D]{e}{\vp}
    \\ n > 0
    \\ K:\ov{\s_i \to_\pi} \s \in \G
    }
    {\judg{K~\ov{\xo},\ov{y_i{:}_1\s_i}^n\to e}{\s \Longrightarrow \vp}[alt][\D_s][z]}
}

\newcommand{\TypeAltN}{
    \mprset{flushleft}
    \infer[(AltN)]
    { \judg[\G,\ov{\xo},\ov{y_i{:}_{\D_i}\s_i}][\D]{e}{\vp}
    \\ \ov{\D_i} = \ov{\lctag{\D_s}{K_j}}^n\\ \ov{\D_i \neq \cdot}
    \\ n > 0
    \\ K:\ov{\s_i \to_\pi} \s \in \G
    }
    {\judg{K~\ov{\xo},\ov{y_i{:}_1\s_i}^n\to e}{\s \Longrightarrow \vp}[alt][\D_s][z]}
}

\newcommand{\TypeAltZero}{
    \mprset{flushleft}
    \infer[(Alt0)]
    { \judg[\subst{\G}{\cdot}{\D_s}_z,\ov{\xo}][\subst{\D}{\cdot}{\D_s}]{e}{\vp}
    \\ K:\ov{\s_i \to_\omega} \s \in \G
    }
    {\judg{K~\ov{\xo}\to e}{\s \Longrightarrow \vp}[alt][\D_s][z]}
}

\newcommand{\TypeAltWild}{
    \mprset{flushleft}
    \infer[(Alt\_)]
    { \judg{e}{\vp} }
    {\judg{\_ \to e}{\s \Longrightarrow \vp}[alt][\D_s][z]}
}

\begin{figure}[h]
\begin{framed}
\small
% TODO: Consider moving the Variable cases to the middle or last place?
\[
\begin{array}{c}
    \judgment{\judg{e}{\s}}
\\[1em]
    \TypeMultLamIntro
\qquad
    \TypeMultLamElim
\\[1em]
    \TypeLamIntroL
\qquad
    \TypeLamIntroW
\\[1em]
    \TypeVarDelta
\qquad
    \TypeVarSplit
% TODO: Somewhere here we need at least the split
\\[1em]
    \TypeVarOmega
\qquad
    \TypeLamElimL
\\[1em]
    \TypeLinearVar
\qquad
    \TypeLamElimW
\\[1em]
    \TypeLet
\qquad
    \TypeLetRec
\\[1em]
    \TypeCaseWHNF
\\[1em]
    \TypeCaseNotWHNF
% Ouch, nope.
% \\[1em]
%     \TypeCaseVar
\\[2em]
    \judgment{\judg{\rho \to e}{\sigma \Rightarrow \varphi}[alt][\Delta_s][z]}
\\[1em]
    \TypeAltN
\\[1em]
    \TypeAltZero
\qquad
    \TypeAltWild
\\[1em]
    \judgment{\Gamma \vdash_{mult} \pi}
\\[1em]
    \infer*[right=$(1)$]
    { }
    {\Gamma \vdash 1}
\qquad
    \infer*[right=$(\omega)$]
    { }
    {\Gamma \vdash \omega}
\qquad
    \infer*[right=$(\rho)$]
    { }
    {\Gamma, \rho \vdash \rho}
\\[1em]
\begin{array}{cc}
\judgment{\Gamma \vdash decl : \Gamma'} & \judgment{\Gamma \vdash pgm : \sigma}\\
\\[0.05em]
\infer*[right=$(Data)$]{ }{\Gamma \vdash (\datatype{T~\overline{p}}{\overline{K:\sigma}}) : (\overline{K:\sigma}) } &
\infer*[right=$(Pgm)$]{\overline{\Gamma \vdash decl:\Gamma_d} \and \Gamma = \Gamma_0,\overline{\Gamma_d}\\\\ \Gamma~\mathsf{is~consistent?} \and \Gamma \vdash e : \sigma}{\Gamma_0 \vdash \overline{decl}; e : \sigma}
\end{array}
\end{array}
\]
\end{framed}
\caption{Linear Core* Typing Rules}
\label{linear-core-typing-rules}
\end{figure}

