\documentclass[]{lwnovathesis}

\usepackage{todonotes}
\usepackage{cmll}
\usepackage{amssymb}
\usepackage{amsmath}
\usepackage{mathpartir}
\usepackage[ruled,vlined]{algorithm2e}
\usepackage{hyperref}
\usepackage{fancyvrb}



\DefineVerbatimEnvironment{code}{Verbatim}{fontsize=\small}
\DefineVerbatimEnvironment{example}{Verbatim}{fontsize=\small}

\newcommand{\mypara}[1]{\paragraph{\textbf{#1}.}}
\newcommand{\lolli}{\multimap}
\newcommand{\tensor}{\otimes}
\newcommand{\one}{\mathbf{1}}
\newcommand{\bang}{{!}}

\newcommand{\llet}[2]{\mathsf{let}~#1~\mathsf{in}~#2}
\newcommand{\ccase}[2]{\mathsf{case}~#1~\mathsf{of}~#2}

\title{Linting Linearity in Core/System FC}
\author{Rodrigo Mesquita}


\begin{document}

\frontmatter

\maketitle
\tableofcontents

\mainmatter

% TODO: linear functions to allow safe/controlled use of reified language
% implementation objects like *single-use continuations* (single use
% continuations? can we make them faster?)

\chapter{Introduction}

% Statically typed languages 
% Linear types are cool
% Among the few, Haskell has linear types
% But Haskell differs from other languages with linear types
%   * added 31 years after inception
%   * added to its core
% Core has linear types for good reasons
% But they are broken -- why?
%   * Optimizations changes source -- e.g. if recursive lets are introduced, how to type linearity?
% Goals - O que eu vou fazer
% Document Structure (pode-se eventualmente combinar com a seccao dos goals)


%\section{Introduction}

Static type systems have brought programmers ...
Programming languages rely on type systems to statically eliminate classes of
errors ...

Linear type systems~\cite{} add expressiveness to existing type systems by
stating that linear resources must be used \emph{exactly once}. In programming
languages with a linear type system, not using certain resources or using them
twice are type errors. Linear types can thus be used to, for example, guarantee
that file handles, socket descriptors, and allocated memory are freed exactly
once (leaks and double-frees become type errors)~\cite{}, implement safe
high-performance language interoperability~\cite{}, guarantee that quantum
entangled states are not duplicated~\cite{}, handle mutable state safely~\cite{},
something on session types guarantees~\cite{}, and much more.

% TODO: Also include bit about how it makes the basis of other type systems such
% as rusts owneship. How ?

As an example, consider the following C-like program in which allocated memory
is freed twice. We know this to be the dreaded double-free error which will
crash our programs and crush our dreams. Regardless, a C-like type system will
accept this program without any issue.
\begin{code}
let p = malloc 4;
 in free p;
    free p;
\end{code}

Under the lens of a linear type system, consider the variable $p$ to be a linear
resource created by the \textbf{malloc} keyword. Because $p$ is a linear
variable, it must be used exactly once. However, we are using it in two
different calls to \textbf{free}. With our new linearity assumptions, the
% TODO:
example program \emph{would not typecheck}! Some conclusion on linearity
goodness. In section~\ref{linear-types} we describe the fundamentals of linear
types in more detail with a formal definition and many examples.

Despite their great promise and research literature, it is hard to balance
linear and non-linear types in practice and, consequently, few programming
languages have linear type systems. Among them are Idris2\cite{}, a linearly and
dependently typed language based on Quantitative Type Theory, Rust\cite{}, a
language whose ownership types build on linear ones to guarantee memory safety
without garbage collection or reference counting, and, more recently,
Haskell\cite{}, a \emph{purely functional} and \emph{lazy} language.

Linearity in Haskell stands out % from linearity in Rust and Idris2
for the two following distinct reasons:

\begin{itemize}
    \item Linear types were only added to the language roughly \emph{31 years
        after} Haskell's inception, unlike Rust and Idris2 which were
        designed with linearity from the start. It is an especially difficult
        endeavour to add linear types to a well-established language with a
        large library ecosystem and many active users, rather than to develop
        the language from the ground up with linear types in mind, and the
        successful approach as implemented in GHC 9.0, the leading Haskell
        compiler, was based on Linear Haskell\cite{linearhaskell}, where a
        linear type system designed with retaining backwards-compatibility and
        allowing code reuse across linear and non-linear users of the same
        libraries in mind was described. We describe Linear Haskell in detail in
        section~\ref{linear-haskell-section}.

    \item Linear types permeate Haskell down to (its) \textbf{Core}, the
        intermediate language Haskell is desugared to. \textbf{Core} is a very
        small explicitly typed functional language to which multiple
        Core-to-Core optimizing transformations are applied in succession. The
        final Core program can be typechecked very fast (because of the explicit
        type annotations) and doing so serves as a sanity check to the
        correction of the source transformations. If the resulting optimized
        Core program doesn't typecheck, something \emph{very wrong} has happened
        in a transformation!  We discuss Core in detail in
        section~\ref{core-section}.
        % TODO: \item values in rust are linear by default while non-linear is
        % the haskell default?
\end{itemize}

Core was imbued with linear types to guarantee linear programs remain linear after being
desugared and optimized: much the same way a \emph{typed} Core ensures that
desugaring, and optimizing transformations, are correctly implemented and result
in a \emph{well-typed} program -- a design decision that has well served the
compiler developers with a safety net for a long time. It is crucial that a
program behaviour enforced by linear types, regarding the usage of certain
resources, is \emph{not} changed in the desugaring or optimization process
(optimizations should not destroy linearity!) and a linearity aware Core
typechecker can guarantee that.

Additionally, while not yet a reality, linearity in Core could be used to inform
certain program optimizations, i.e. having linear types in Core could be used to
further optimize certain programs and, therefore, benefit the runtime
performance characteristics of our programs. For example, Linear Haskell\cite{}
describes as future work an improvement to the inlining optimization: Inlining
is a centerpiece program optimization primarily because of the subsequent
optimizing opportunities unlocked by inlining. However, it relies on a heuristic
process known as \emph{cardinality analysis} to discover safe inlining
opportunities. Unfortunately, heuristics can be volatile and fail in identifying
crucial inlining opportunities resulting in programs that fall short of their
performance potential. On the contrary, the linearity information could be
integrated in the analysis and used as a (non-heuristic) cardinality
\emph{declaration}.

However, currently, programs transformed by optimizations can fail to typecheck
in Core because of the newly added linearity! The reason is not evidently clear:
if we can typecheck linearity in the surface level Haskell why can't we in Core?
In fact, if we were to typecheck the programs in Core as they were at the
surface level, we would succeed, but programs are \emph{transformed} (by
semantic-preserving optimizing transformations) and hence end up different from
how they were originally written. For example, the following recursive let
definition, where $x$ is a linear variable that must be used exactly once, would
not typecheck in a source Haskell program since the typechecker cannot tell that
$x$ is used linearly, but this program might occur naturally in Core from
transformations on a program that did succesfully typecheck:
\begin{code}
letrec f = \case
        True -> f False
        False -> x
in f True
\end{code}

Is this program really still linear? Yes, but ...

% If one looks at it 

% Alternatively, one might imagine Haskell being desugared into an intermediate
% representation to which multiple different optimizing transformations are
% applied but on which no integrity checks are done

% Despite \emph{want} a linearly-typed core
% because a linearly-typed Core ensures that desugaring
% Haskell and optimizing transformations don't destroy linearity.

% In theory, because the Core language must also know about linearity, we should

% A remedy is to use the multiplicity annotations of λq → as cardinality declarations. Formalising
% and implementing the integration of multiplicities in the cardinality analysis is left as future work.

% In theory, we should typecheck \emph{linearity} in \textbf{Core} just the same
% as the typechecking verification we had with the existing type annotations prior
% to the addition of linear types, that is, our Core program with linearity
% annotations should be typechecked after the optimizing transformations...

% Even though Linear Haskell was successful in integrating linear types in an
% existing language 
% In spite of the smooth integration of linear types in an existing language
% with regard to backwards compatibility and (re)usability, 

% The advent of linear types in Haskell bring forth

% Besides Haskell's supporting linear
% types according to the said paper, Idris2\cite{} supports linear types in a
% dependently typed setting, Clean\cite{} has uniqueness types which are closely
% related to linear types, and Rust\cite{} has ownership types which build from
% linear types. 
% released in the Glasgow Haskell Compiler (GHC) version 9.0.
% , and, e.g., avoid the required boilerplate threading of linear resources that
% we will get to know ahead.

% Linear types were introduced in GHC 9.0


% In an attempt to bridge the gap between the theory and practicality of linear
% types, Linear Haskell\cite{} describes , the 9.0 version series of Haskell's
% \emph{de facto} compiler, GHC, supports linear types. However, retrofitting
% linear types to a purely-functional lazy language 

% The main contributions of this thesis are:
% \begin{itemize}
%     \item Core Type system which accepts optimized linear programs
% \end{itemize}

% stores a value to the allocated memory, reads from it and finally frees 

% about the usage of resources in a
% programming language.

\section{Motivation}


\begin{itemize}
    \item Are we really preserving linearity?
    \item Inform/unlock other optimisations that take into account linearity
\end{itemize}

% TODO? Another defining feature of Haskell is its powerful type system. The
% interaction between the new linearity polymorphism and the existing
% type-level facilities is quite interesting

\chapter{Background}

\section{Linear Types\label{linear-types}}

Much the same way type systems can statically eliminate various kinds of
programs that would fail at runtime, such as a program that derreferences an
integer value rather than a pointer, linear type systems can guarantee that
certain errors (regarding resource usage) are forbidden.

In linear type systems, so called linear resources must be used \emph{exactly
once}. Not using a linear resource at all or using said resource multiple times
will result in a type error. We can model many real-world resources such as file
handles, socket descriptors, and allocated memory, as linear resources. This
way, because a file handle must be used exactly once, forgetting to close the
file handle is a type error, and closing the handle twice is also a type error.
With linear types, avoiding leaks and double frees is no longer a programmer's
worry because the compiler can guarantee the resource is used exactly once, or
\emph{linearly}.

To understand how linear types are defined and used in practice, we present two
examples of anonymous functions that receive as an argument a handle to work
with and are responsible for closing it, we try to understand how the examples
could be disregarded as incorrect, and work our way up to linear types from
them. The first function ignores the received file handle and returns $\star$
(read unit), which is equivalent to C's \textbf{void}.
% TODO: Better code formatting
\begin{mathpar}
    \lambda h.~\textrm{return}~\star;
    \and
    \lambda h.~\textrm{close}~h;~\textrm{close}~h;
\end{mathpar}
% TODO: Equation references?

Ignoring the file handle which should have been closed by the function makes the
first function incorrect. Similarly, the second function receives the file handle and
closes it twice, which is incorrect not because it falls short of the
specification, but rather because the program will crash while executing it.
%
Additionally, both functions share the same type, $\textbf{Handle} \to \star$, i.e.
a function that takes a \textbf{Handle} and returns $\star$. The second function also
shares this type because \textbf{close} has type $\textbf{Handle} \to \star$.
%
Under a simple type system such as C's both functions are type correct (the
compiler will happily succeed), but both have erroneous behaviours. The first
forgets to close the handle and the second closes it twice. Our goal is now to
reach a type system that rejects these two programs.

The key observation to invalidating these programs is to focus on the function
type going between \textbf{Handle} and $\star$ and augment it to indicate that
\emph{the argument must be used exactly once}, or, in other words, that the
argument of the function must be linear. We take the function type $A \to B$
and replace the function arrow ($\to$) with the linear function arrow ($\lolli$)
%
\footnote{Since linear types are born from a correspondence with linear
logic\cite{linear-logic} (the Curry-Howard isomorphism\cite{curry-howard}), we
borrow the $\lolli$ symbol and other linear logic connectives to describe linear
types.}
%
operator to denote a function that uses its argument exactly once: $A \lolli B$.
%
Providing the more restrictive linear function signature $\textbf{Handle} \lolli
\star$ to the example programs would make both of them fail to typecheck because
they do not satisfy the linearity specification that the function argument
should only be used exactly once.

% Linear types are a powerful idea because they allow us to statically reason
% about resources in our program. A so called linear resource ?

In order to further give well defined semantics to a linear type system, we
present a linearly typed lambda calculus, a very simple language with linear
types, by defining what are syntatically valid programs through the grammar and
what programs are well typed through the typing rules.

An expression in the linearly typed lambda calculus can be $\star$...
A type in the linearly typed lambda calculus can be $\star$...

\begin{mathpar}
  \begin{array}{lcl}
    s,t & ::= & \star\\
        & \mid & t \lolli s\\
        & \mid & t \oplus s\\
        & \mid & t \tensor s\\
        & \mid & t \with s\\
        & \mid & \bang t\\
   \end{array}

\and

  \begin{array}{lcl}
      M,N & ::= & \star \mid \llet{\star = M}{N} \\
        & \mid & u \\
        & \mid & \lambda u. M \mid M~N\\
        & \mid & \textrm{inl}~M \mid \textrm{inr}~M \mid
        \ccase{M}{\textrm{inl}~u_1 \to N_1 \mid \textrm{inr}~u_2 \to N_2}\\
        & \mid & M \tensor N \mid \llet{u_1 \tensor u_2 = M}{N}\\
        & \mid & M \with N \mid \textrm{fst}~M \mid \textrm{snd}~M \\
        & \mid & \bang M \mid \llet{!u = M}{N} \\
   \end{array}
\end{mathpar}

\begin{mathpar}
    \infer*[right=($\tensor I$)]
    {\Delta \vdash M : A \and \Delta' \vdash N : B}
    {\Delta , \Delta' \vdash (M \tensor N) : A \tensor B}

    \and
\end{mathpar}

\begin{mathpar}
    \infer*[right=($\tensor E$)]
    {\Delta \vdash M : A \tensor B \and \Delta', u{:}A, v{:}B \vdash N : C}
    {\Delta , \Delta' \vdash \llet{u \tensor v}{N} : C}
\end{mathpar}

\begin{mathpar}
    \infer*[right=($\lolli I$)]
    {\Delta , u{:}A \vdash M : B}
    {\Delta \vdash \lambda u. M : A \lolli B}

    \and

    \infer*[right=($\lolli E$)]
    {\Delta \vdash M : A \lolli B \and \Delta' \vdash N : A}
    {\Delta, \Delta' \vdash M~N : B}
\end{mathpar}

\begin{mathpar}
    \infer*[right=($\with I$)]
    {\Delta \vdash M : A \and \Delta \vdash N : B}
    {\Delta \vdash M \with N : A \with B}

    \and

    \infer*[right=($\with E_L$)]
    {\Delta \vdash M : A \with B}
    {\Delta \vdash \textrm{fst}~M : A}
    
    \and

    \infer*[right=($\with E_R$)]
    {\Delta \vdash M : A \with B}
    {\Delta \vdash \textrm{snd}~M : B}
\end{mathpar}

\begin{mathpar}
    \infer*[right=($\oplus I_L$)]
    {\Delta \vdash M : A}
    {\Delta \vdash \textrm{inl}~M : A \oplus B}

    \and

    \infer*[right=($\oplus I_R$)]
    {\Delta \vdash M : B}
    {\Delta \vdash \textrm{inr}~M : A \oplus B}

    \and

    \infer*[right=($\oplus E$)]
    {\Delta \vdash M : A \oplus B \and \Delta', w_1{:}A \vdash N_1 : C \and
    \Delta', w_2{:}B \vdash N_2 : C}
    {\Delta, \Delta' \vdash \ccase{M}{\textrm{inl}~w_1 \to N_1 \mid \textrm{inr}~w_2 \to N_2} : C}
\end{mathpar}

\begin{mathpar}
    \infer*[right=($\star I$)]
    { }
    {\Delta \vdash \star : \star}

    \and

    \infer*[right=($\star E$)]
    {\Delta \vdash M : \star \and \Delta' \vdash N : B}
    {\Delta, \Delta' \vdash \llet{\star = M}{N} : B}

\end{mathpar}

\begin{mathpar}
    \infer*[right=($\bang I$)]
    {\Gamma ; \cdot \vdash M : A}
    {\Gamma ; \cdot \vdash !M : !A}

    \and

    \infer*[right=($\bang E$)]
    {\Gamma ; \Delta \vdash M : \bang A \and \Gamma, u{:}A ; \Delta' \vdash N : C}
    {\Gamma ; \Delta, \Delta' \vdash \llet{!u = M}{N} : C}
\end{mathpar}

% A linear type system is a type system in which linear resources must be used
% exactly once...!

% % A linear type system is a type system in which statements such as "this function
% % will use its argument exactly once" can be formally expressed and checked.

% Girard's \emph{linear logic} suggests a new \emph{type system} for functional
% languages through the Curry-Howard isomorphism -- every linear logic proposition
% maps to a different linear type.

% Values belonging to a linear type must be used exactly once. For example, the
% type corresponding to the linear implication ($\lolli$), the linear function, is
% a function that consumes its argument exactly once. A linear function that
% receives an integer as an input and consumes it exactly once to
% produce a character would have type $\textbf{int} \lolli
% \textbf{char}$.

% Programming languages with linear type systems might differ in their definition of linear
% resources, but usually \emph{variables} are considered linear resources?
% % That is, we can rely on the type system to prove properties regarding the usage of certain resources.
% % The \emph{Hello World} of linearly typed programs is a linearly typed pointer.
% % In a non-memory managed language, an allocated pointer must be freed
% % \emph{exactly once}.

% % \begin{code}
% % let p = malloc(5)
% %     p' = store(p, "x")
% %     (p'', v) = load(p)
% %     free p''
% % \end{code}

% % We'd like to hide the boilerplate involved \cite{Linear constraints}


% \mypara{Type Systems} A type system can be formally described through
% a set of inference rules that inductively define a judgment of the
% form $\Gamma \vdash M : A$, stating that program expression $M$ has
% type $A$ according to the \emph{typing assumptions} for variables
% tracked in $\Gamma$. For instance,
% $x{:}\mathsf{Int}, y{:}\mathsf{Int} \vdash x+y : \mathsf{Int}$ states
% that $x+y$ has type $\mathsf{Int}$ under the assumption that $x$ and
% $y$ have type $\mathsf{Int}$.  An expression $M$ is deemed well-typed
% with a given type $A$ if one can construct a typing derivation with $M :
% A$ as its conclusion, by repeated application of the inference rules.

% The simply-typed $\lambda$-calculus is a typed core functional
% language~\cite{10.5555/509043} that captures the essence of a type system in a simple and familiar environment. Its syntax consists of
% functional abstraction, written $\lambda x{:}A.M$, denoting
% an (anonymous) function that takes an argument of type $A$, bound to
% $x$ in $M$; and application $M\,N$, with the standard meaning, and
% variables $x$. For instance, the term $\lambda x{:}A.x$, denoting the identity function, is a functional abstraction
% taking an argument of type $A$ and returning it back.

% \mypara{Linear Logic}

% Linear logic \cite{DBLP:journals/tcs/Girard87} can be seen as a
% resource-aware logic, where propositions are interpreted as resources
% that are consumed during the inference process.  Where in standard
% propositional logic we are able to use an assumption as many times as
% we want, in linear logic every resource (i.e., every assumption) must
% be used \emph{exactly once}, or \emph{linearly}. This usage
% restriction gives rise to new logical connectives, based on the way
% the ambient resources are used. For instance, conjunction, usually
% written as $A\wedge B$, appears in two forms in linear logic:
% multiplicative or simultaneous conjunction (written $A\tensor B$); and
% additive or alternative conjunction (written $A\with
% B$). Multiplicative conjunction denotes the simultaneous availability
% of resources $A$ and $B$, requiring both of them to be
% used. Alternative conjunction denotes the availability of $A$ and $B$,
% but where only one of the two resources may be used. Similarly,
% implication becomes linear implication, written $A\lolli B$,
% denoting a resource that will consume (exactly one) resource $A$ to
% produce a resource $B$.

% To present the formalization of this logic, besides the new
% connectives, we need to introduce the \emph{resource-aware context}
% $\Delta$.  In contrast to the previously seen $\Gamma$, $\Delta$ is
% also a list of variables and their types, but where each and every
% variable must be used exactly once during inference.  So, to introduce
% the connective $\tensor$ which defines a multiplicative pair of
% propositions, we must use exactly all the resources
% ($\Delta_1, \Delta_2$) needed to realize the ($\Delta_1$)
% proposition $A$, and ($\Delta_2$) proposition $B$:
% \[
%     \infer*[right=($\tensor I$)]
%     {\Delta_1 \vdash M : A \and \Delta_2 \vdash N : B}
%     {\Delta_1 , \Delta_2 \vdash (M \tensor N) : A \tensor B}
% \]
% Out of the logical connectives, we need to mention one more, since it
% augments the form of the judgment and it's the one that ensures
% logical strength i.e. we're able to translate intuitionistic logic
% into linear logic.  The proposition $\bang A$ (read \emph{of course}
% $A$) is used (under certain conditions) to make a resource
% ``infinite'' i.e. to make it useable an arbitrary number of times. To
% distinguish the ``infinite'' variables, a separate, unrestricted,
% context is used -- $\Gamma$. So $\Gamma$ holds the ``infinite''
% resources, and $\Delta$ the resources that can only be used once.  The
% linear typing judgment for the introduction of the exponential $\bang A$
% takes the form:
% \[
%     \infer*[right=($\bang I$)]
%     {\Gamma ; \emptyset \vdash M : A}
%     {\Gamma ; \emptyset \vdash \bang M : \bang A}
% \]
% Logically, a proof of $\bang A$ cannot use linear resources since
% $\bang A$ denotes an unbounded (potentially $0$) number of copies of
% $A$. Proofs of $\bang A$ may use other unrestricted or exponential
% resources, tracked by context $\Gamma$.
% From a computational perspective, the type $\bang A$
% internalizes the simply-typed $\lambda$-calculus in the linear
% $\lambda$-calculus.

% The elimination form for the exponential, written $\llet{ !u
% = M}{ N}$, warrants the use of resource $A$ an unbounded
% number of times in $N$ via the variable $u$:
% \[
%     \infer*[right=($\bang E$)]
%     {\Gamma ; \Delta_1 \vdash M : \bang A \and \Gamma, u{:}A ; \Delta_2 \vdash N : C}
%     {\Gamma ; \Delta_1, \Delta_2 \vdash \llet{ !u = M}{ N} : C}
% \]


% % TODO: Revise the story above instead of dumping so much information



% \mypara{Linear Types}
% \subsection{Linear Types}

% Girard's \emph{linear logic} suggests a new \emph{type system} for functional
% languages through the Curry-Howard isomorphism -- every linear logic proposition
% maps to a different linear type.

% Values belonging to a linear type must be used exactly once. For example, the
% type corresponding to the linear implication ($\lolli$), the linear function, is
% a function that consumes its argument exactly once. A linear function that
% receives an integer as an input and consumes it exactly once to
% produce a character would have type $\textbf{int} \lolli
% \textbf{char}$.


% \begin{itemize}
%     \item Ideia dos linear types
%     \item Calculo lambda linear (como forma de falar da linearidade)
%     \item Gestão de recursos
%     \item Rust (core is linear lambda calculus?)
% \end{itemize}

\section{Haskell}

Haskell is a functional programming language defined by the "Haskell
Report"\cite{jones1999haskell,marlow2010haskell} and whose \emph{de-facto}
implementation is GHC, the Glasgow Haskell Compiler\cite{GHC}. Haskell is a
\emph{lazy}, \emph{purely functional} language, i.e., functions cannot
have side effects or mutate data, and contrary to many programming languages,
arguments are \emph{not} evaluated when passed to functions, but rather are only
evaluated when the value is needed. The combination of purety and laziness is
unique to Haskell among mainstream programming languages...

Haskell is a large language but its small core is based on a typed lambda
calculus. As such, there exist no statements and computation is done simply
through the evaluation of functions. Besides functions, one can define types and
their constructors and pattern match on said constructors. Function application
is denoted by an empty space (\textbf{f~a} means \textbf{f} applied to
\textbf{a}) and pattern matching is done with the \textbf{case} keyword followed
by the enumerated alternatives. All variable names start with lower case and
types start with upper case (excluding type variables). To make explicit the
type of an expression, the \textbf{::} operator is used (e.g.
$\textbf{f~::~Int}\to \textbf{Bool}$ is read \textbf{f} \emph{has type} function
from \textbf{Int} to \textbf{Bool}).

Because Haskell is a pure programming language, input/ouptut side-effects are
"made pure" by modelling them on the type level through the
higher-kinded\footnote{\textbf{IO} has kind $\textbf{Type}\to\textbf{Type}$,
that is, it is only a type after another type is passed as a parameter (e.g.
\textbf{IO~Int}, \textbf{IO~Bool}); \textbf{IO} by itself is a \emph{type
constructor}} type \textbf{IO}. Some of the example programs will look though as
if they had statements, but, in reality, the sequential appearence is just
syntatic sugar to an expression using Monadic operators. The main take away is
that computations that do I/O may be sequenced together with other operations
that do I/O while retaining the lack of statements and the language purity
guarantees.

% TODO: Perhaps elaborate a bit more on the constructors and pattern matching?

As an example, consider these functions that do I/O and their types. The first
opens a file by path and returns its handle, the second gets the size of a file
from its handle, and the third closes the handle. It is important that the
handle be closed exactly once, but currently nothing enforces that usage policy.

\begin{code}
openFile :: FilePath -> IOMode -> IO Handle
hFileSize :: Handle -> IO Integer
hClose   :: Handle -> IO ()
\end{code}

The following function makes use of the above definitions to return the size of
a file given its path.
\begin{code}
countWords :: FilePath -> IO Integer
countWords path = do
    handle <- openFile path ReadMode
    size   <- hFileSize handle
    return size
\end{code}
% hClose handle

Another defining feature of Haskell is its powerful type system....


% TODO: File example

A function to compute the length of a list in Haskell is defined by pattern
matching on the list and returning $0$ if the list is empty and adding one to
the (computed recursively) length of the rest of the list otherwise. The
following snippet defines the said \textbf{length} function.
\begin{code}
    length [] = 0
    length (x:xs) = 1 + length xs
\end{code}
The function \textbf{length} receives a list and returns an integer, and since
it treats all list's elements the same, it is said to be polymorphic over the
list's elements type. In Haskell, function signatures typically appear above the
definition, and for this particular function we'd have
\begin{code}
    length :: [a] -> Int
\end{code}
where \textbf{a} is a universally quantified type variable, that is,
\textbf{length} can be called on lists of ints, lists of characters, lists of
lists of strings, lists of anything.

The laziness evaluation strategy entails expressions are only evaluated when
needed. If we were to call just the above function on a list of computationally
expensive elements, such as
\begin{code}
    length [5^1234, 6^4567, 8^(9^42)]
\end{code}
We would get the result 3 instantaneously: the elements of the list would never
be evaluated because the \textbf{length} function does not require the value of
each element to compute a result.

% TODO:
% Is it outside the scope of this report to discuss the merits and disadvantages
% of laziness and purely functional properties?


\begin{itemize}
    \item Features fundamentais
    \item Exemplos, introduzir syntax
    \item (Ao contrário de OCaml) tem um sistema de tipos muito sufesticado
    \item Eventualmente explicação de GADTs? c/ exemplo do Vec
    \item Uma das features mais avaçadas/experimentais é o Linear Haskell
\end{itemize}

\subsection{Linear Haskell\label{linear-haskell-section}}

\begin{itemize}
    \item Linear Haskell definition
    \item Consuming values precisely
    \item Multiplicities
    \item Multiplicity polymorphism
    \item Que relaciona o systemFC com o calculo lambda linear
\end{itemize}

\subsection{Core Haskell\label{core-section}}

\begin{itemize}
    \item What is Core (IR) e as particularidades
    \item Em cima do SystemFC
    \item Optimizações feitas no Core
    \item Base SystemFC tem de ser "merged" com o linear lambda calculus
    \item Ou porque é que é importante o Core ter linearidade também/a que nível
        foi introduzido
    \item Falar das coercions em especifico
    \item E das local equalities que são implicitas no front end
\end{itemize}

\section{GHC Pipeline}

\begin{itemize}
    \item Parser -> Rename -> Typechecker -> Desugar
    \item Core2Core transformations
    \item GHC unique em haver tantas transformações sempre sobre a mesma intermediate
        representation como input e output
    \item 
\end{itemize}

% Previous itemize. Talvez ainda tenha de ter secções sobre os últimos dois
% pontos
% \begin{itemize}
%     \item Linear types
%     \item Linear Haskell
%     \item Core
%     \item Sistemas de inferência
%     \item GADTs e Coercions
% \end{itemize}

\chapter{Related Work}

\section{System FC}

\section{Linear Haskell}

\section{OutsideIn(X)}

Se for modificar o typechecker com as multiplicity coercions vou ter de falar
disto.

\section{Rust}

They have a linearly typed something?

\chapter{Technical Details}

% \section{Foreword}

% This document is a work in progress proposal that I will deliver (in an extended
% form) at the end of the semester as part of my master thesis preparation. The
% document still needs a proper introduction, lengthy background and related work
% section  such that an outsider might understand what I propose to do. However,
% this initial iteration is instead targeted at those already familiar with the
% problem and serves as a (less bureaucratic) proposal on linting linearity in
% Core and to showcase the preliminary progress I have made so far.


\section{Introduction}


Since the publication of Linear Haskell~\cite{linearhaskell} and its release in GHC 9.0,
Haskell's type system supports linearity annotations in functions -- bringing
linear types to a mainstream, pure, and lazy functional language. Concretely,
function types can be annotated with a multiplicity (a multiplicity of One
requires the argument to be consumed exactly once, a multiplicity of Many allows
the argument to be consumed unrestrictedly, i.e., zero or more times). A
function is linear if it consumes its arguments exactly once when it itself is
consumed exactly once.
% TODO: for some definition of \emph{consume} that I should revise here.

Linearly typed programs are proven/checked to be linear by the type system, so
the addition of linear types evidently required changing the type checker to
support linearity. There exist, however, two distinct type checkers in GHC.
The first is run on the surface language, i.e. the Haskell we write, and is a
complex type checker that now also supports typing linearity. The second is
run on programs written in the intermediate language \emph{Core}, that
are obtained from \emph{desugaring} Haskell.

Core is a much smaller language than the whole of Haskell (even though we
can compile the whole of Haskell to it!) and its type checker is
simple and fast due to Core being explicitly typed. In essence, Core
is close to a higher-order polymorphic lambda calculus. This type
checker (called \emph{Lint}) gives us guarantees of correctness in face of all
the complex transformations a Haskell program undergoes, such as desugaring and
core-to-core optimization passes, because the linter is always run on the resulting code
before ultimately being compiled to (untyped) machine code.

% System FC is the formal system in which the implementation of GHC's intermediate
% representation language \emph{Core} is based on.

We want Core and its type system to give us guarantees about
desugaring and transformations with regard to linearity -- a linearly
typed Core ensures that linearly-typed programs remain correct
(i.e.,~linearity is preserved) after desugaring and all GHC
transformations (i.e.,~optimisations should not destroy linearity).

In this sense, Core is already annotated with linearity, but the \textbf{linter currently
  ignores linearity annotations}.
%
In spite of the strong formal foundations of linear types driving the
implementation, their interaction with the whole of GHC is still far
from trivial. The implemented type system cannot accomodate
several optimising transformations that produce programs that violate
linearity syntactically (i.e.,~multiple occurrences of linear
variables in the program text), but ultimately preserve it in a
semantic sense, where a linear term is still consumed exactly once --
this is compounded by lazy evaluation driving a non-trivial mismatch
between syntactic and semantic linearity. Therefore, linear linting
rejects various valid programs (with regard to linearity) after
desugaring. The current solution to this is to effectively disable the
linear linter, since otherwise disabling optimisations can incur significant
performance costs. However, we believe that GHC's transformations are
correct, and it is the linear linter and its underlying type system
that cannot sufficiently accommodate the
resulting programs.
% diverse problems with linearity spring up when we get past the desugarer and
% the
% linter rejects valid programs which is quite undesirable.

% For example, optimizing transformations, coercions and type families, recursive
% lets, and empty case expressions don't currently fit in with linearity.

We propose to extend Core's type system with additional linearity
information in order to accommodate linear programs in Core that
result from optimising transformations. We will prove the system's
soundness/type safety and show how it validates some core-to-core optimisation
passes. We also plan to implement the extension into GHC's
linter. Concretely, we will extend Core's linear type system with
\emph{usage annotations} for let, letrec and case binder bound
variables. Additionally, we will explore a new kind of coercion for
multiplicities to validate programs that combine GADTs with
multiplicities.
%
The ultimate measure of success is the \verb=-dlinear-core-lint= flag,
which activates the linear linter. In its current implementation,
enabling this flag rejects many linearly valid programs. Ideally, by
the end of our research and implementation, this flag could be enabled by
default and accommodate all existing transformations. Realistically, we want to
accept as many diverse transformations as possible while still preserving
linearity, even if we are unable to account for all of them.

In Section~\ref{typingUsageEnvs} we describe usage environments and how they
solve three distinct problems. In Section~\ref{multiplicityCoercions} we discuss
the beginning of multiplicity coercions. In Section~\ref{examples} we take
programs that are currently rejected and show how the type system with our
extensions can accept them.

% to be able to preserve linearity accross the stages and to enable \emph{Lint} to
% preserve our sanity regarding linearity and eventually inform with linearity new
% transformations.

% A value allocated to be passed to a linear function and then never again used could
% bypass garbage collection.

% Our key innovation is that, by recognising join points as a language construct,
% we both preserve join points through subsequent transformations and exploit them
% to make those transformations more effective. Next, we formalize this ap-
% proach; subsequent sections develop the consequences.

\section{Typing Usage Environments\label{typingUsageEnvs}}

Haskell has call-by-need semantics that entail that an expression is not
evaluated when it is let-bound but rather when the binding variable is used.
This makes it challenging to reason about linearity for let-bound variables. The
following example fails to typecheck in Haskell, but semantically it is indeed
linear due to lazy evaluation:

\begin{code}
f :: a %1 -> a
f x = let y = x in y
\end{code}

Despite not being accepted by the surface-level language, linear programs using
lets occur naturally in Core due to optimising transformations that create let
bindings. In a similar fashion, programs which violate syntatic linearity
for other reasons other than let bindings are produced by Core transformations.

The solution we found to a handful of problems regarding binders is to have a
\emph{usage environment}. A usage environment is a mapping from variables to
multiplicities. The main idea is to annotate those binders with a usage
annotation rather than a multiplicity (in contrast, a lambda-bound variable has
exactly one multiplicity). When we find a bound variable with a usage
environment, we type linearity as if we were using all variables with
corresponding multiplicities from that usage environment.

In the above example, this would amount to annotating $y$ with a usage
environment $x := 1$ (because the expression bound by $y$ uses $x$ one time).
Upon using $y$, we are actually using $x$ one time, and therefore linearity is
preserved.

% The first set of problems appears in the core-to-core optimisation passes. GHC
% applies many optimising transformations to \emph{Core} and we believe those
% transformations preserve linearity. However, Core's linear type system cannot check
% that they indeed preserve linearity.

% The simpler examples come from straightforward and common optimising
% transformations. Then we have recursive let definitions that don't accommodate
% linearity even though it might converge to only use the value once. Finally, we
% have the empty case expression introduced with the \emph{EmptyCase} language
% extension that Core currently can't typecheck either.

% The key idea is to annotate every variable with either its multiplicity, if it's
% lambda bound, or with its usage environment, if it's let bound.

% A usage environment is a mapping from variables to multiplicities.

% When we lint a core expression, we get both its type and its usage environment.
% That means that to lint linearity in an expression, whenever we come across a
% free variable we compute its usage environment and take it into account

% TODO: When we type an expression we get both the type and usage environment

\subsection{Let}

Let bindings in Core are the first family of problems we tackle with usage
environment annotations. Multiple transformations can introduce let-bindings,
such as CSE and join points. By extending the type system to allow let-bindings in Core
we start paving the way to a linear linter.

Currently, every variable in Core is annotated with a multiplicity at its binding
site. The multiplicity for let-bound variables must be ignored throughout the
transformation pipeline or otherwise too many valid programs would be
rejected for violating linearity in its transformed type.

However, programs with let bindings can be correctly typed by associating a
usage environment to the bound variables. Instead of associating a multiplicity
to every binder, we want to associate a multiplicity if the variable is lambda
bound and a usage environment when it is let-bound. We then instantiate the usage
environment solution to lets in particular -- a let bound variable is annotated
with the usage environment computed from the binder expression; in the let body
expression, when we find an occurrence of the let bound variable we emit its
usage environment. The typing rule is the following:

\begin{mathparpagebreakable}
    \infer*[right=(let)]
    {\Gamma \vdash t : A \leadsto \{U\} \\
     \Gamma ; x :_{U} A \vdash u : C \leadsto \{V\}}
    {\Gamma \vdash \text{let } x :_{U} A = t \text{ in } u : C \leadsto \{V\}}
\end{mathparpagebreakable}

Take for example an expression in which $y$ and $z$ are lambda-bound with a
multiplicity of one. In the following code it might not appear as if $y$
and $z$ are both being consumed linearly, but indeed they are since in the first
branch we use $x$ -- which means using $y$ and $z$ linearly -- and we use $y$
and $z$ directly on the second branch. Note again that let binding $x$ doesn't
consume $y$ and $z$ because of lazy evaluation. Only \emph{using} $x$ consumes $y$ and
$z$.

\begin{code}
let x = (y, z) in
case e of
  Pat1 -> … x …
  Pat2 -> … y … z …
\end{code}

If we annotate the $x$ bound by the let with a usage environment $\delta$
mapping all free variables in its binder to a multiplicity ($\delta = [y := 1, z
:= 1]$), we could, upon finding $x$, simply emit that $y$ and $z$ are consumed
once. When typing the second branch we'd also see that $y$ and $z$ are used
exactly once. Because both branches use $y$ and $z$ linearly, the whole case
expression uses $y$ and $z$ linearly.

% Currently, in GHC, we don't annotate let-bound variables with a usage
% environment, but we already calculate a usage environment and use it to check
% some things (which things?)

% \begin{itemize}
%     \item occurrences? store usage environment in Ids (vars)
%     \item Recursive lets (can it be rewritten using fix)
%     \item Empty case expression
% \end{itemize}

% \subsection{Inlining}

% If we annotate the let bound variables with their usage and emit that usage when
% we come across those variables, we can solve the linearity issues with inlining.

\subsection{Recursive Lets}

A recursive let binds a variable to an expression that might use the bound
variable in its body. Certain optimisations can create a recursive let that uses
linearly bound variables in its binder body. We want to treat recursive lets
similarly to lets: attribute to bound variables a usage environment and emit it
when the variable is used.
% TODO: Replace "certain" with concrete examples

The challenging part about determining the usage environment of the variables bound
by a recursive let is knowing what usage to emit inside their own body, when
computing said usage environment. The example below uses $x$ linearly % \ref{example_letrec}
but how can we prove it? If we are able to somehow compute $f$'s usage
environment to $x := 1$, we know $x$ is used once when $f$ is applied to
$\mathit{True}$.

\begin{code}
letrec f = \case
        True -> f False
        False -> x
in f True
\end{code}

The key idea to computing the usage environment of multiple variables that use
themselves in their definition body is to perform the computation in two separate passes. First,
calculate a naive environment by emiting free variables multiplicities as usual
while treating the recursive-let bound variables as free ones. Second, pass the
variables names and their corresponding naive environment as input to
algorithm~\ref{computeRecUsages} to get a final usage environment. Intuitively,
the algorithm, for each recursive binder, iterates over all (initially naive) usage
environments and substitutes the recursive binder by the corresponding usage
environment, scaled up by the amount of times that recursive binder is used in
the environment being updated.

The algorithm for computing the usage environment of a set of
recursive definitions works as follows. An at least informal proof
that the algorithm is correct should eventually follow this.

\begin{itemize}
    \item Given a list of functions and their naive environment computed from
        the body and including the recursive function names ($(f, F), (g, G),
        (h, H)$ in which there might exist multiple occurrences of $f, g, h$ in $F, G, H$
    \item For each bound rec var and its environment, update all bindings and
        their usage as described in the algorithm
    \item After iterating through all bound rec vars, all usage environments
        will be free of recursive bind usages, and hence "final"
\end{itemize}

Note that the difficulty of calculating a usage environment for $n$
recursive-let bound variables increases quadratically, i.e. the algorithm has
$O(n^2)$ complexity, but this is not a problem since it's rare to have more than
a handful of binders in the same recursive let block.

% TODO: I should probably use the re-computed usageEnvs instead of the naiveUsageEnvs.
% I'm pretty sure it might fail in some inputs if I keep using the naiveUsageEnvs.
\begin{algorithm}
$usageEnvs \gets naiveUsageEnvs.map(fst)$\;
\For{$(bind, U) \in naiveUsageEnvs$}{
    \For{$V \in usageEnvs$}{
        $V \gets sup(V[bind]*U\setminus\{bind\}, V\setminus\{bind\})$
    }
}
\caption{computeRecUsages\label{computeRecUsages}}
\end{algorithm}

Putting the usage environment idea together with the algorithm we obtain the following typing rule:

\begin{mathparpagebreakable}
    \infer*[right=(letrec)]
    {\Gamma ; x_1 : A_1 \dots x_n : A_n \vdash t_i : A_i \leadsto \{U_{i_\text{naive}}\} \\
     (U_1 \dots U_n) = \mathit{computeRecUsages}(U_{1_\text{naive}} \dots U_{n_\text{naive}}) \\
     \Gamma ; x_1 :_{U_1} A_1 \dots x_n :_{U_n} A_n \vdash u : C \leadsto \{V\}}
    {\Gamma \vdash \text{let } x_1 :_{U_1} A_1 = t_1 \dots x_n :_{U_n} A_n = t_n \text{ in } u : C \leadsto \{V\}}
\end{mathparpagebreakable}

Instantiating the usage environment solution according to the above description,
here's an informal proof that the previous example is well typed: We %\ref{example_letrec}
compute the naive usage environment of $f$ to be $x := 1, f := 1$. We compute
its actual usage environment by scaling the naive environment of $f$ without $f$
by the amount of times $f$ appears in the naive environment of $f$ ($1*[x :=
1]$) and get $x := 1$. Finally, we emit $x := 1$ from applying $f$ in the let's
body.

% Following the idea of making let-bound variables remember the usage environment
% here's an informal description that that example is well typed. We want to
% anotate the let-bound variable $f$ with a usage environment delta $\delta$, and
% use the binding body to compute it.

% Description of computing usage environment of a case expression:
% https://gitlab.haskell.org/ghc/ghc/-/wikis/uploads/355cd9a03291a852a518b0cb42f960b4/minicore.pdf.

% However, not understanding it, my take would be that we can compute the usage
% environment of every branch of the case expression and make sure that they all
% unify (wrt to submultiplicities). The special case is when the branch happens to
% call $f$ itself while computing the usage environment of $f$. If it were another
% let bound variable we'd add its own usage environment to the one we're
% computing; if it were a lambda bound variable we'd add [itself $:=$ its
% multiplicity].

% My idea is that we can emit a special usage [$rec := p$], which, when unified
% against the other case branches $\delta$ will always succeed with a unification
% mapping from $rec \rightarrow p\delta$ scaled by the multiplicity of $rec$

% So taking the example, to compute the usage environment of $f$, we'd compute for
% the second branch $[x := 1]$ and for the first branch $[rec := 1]$. Then, we'd
% unify them with $rec \rightarrow [x := 1*1]$, and somehow result in $[x := 1]$

The next example is a linearly invalid program because $x$ is a %~\ref{example_letrec_2}
linearly bound variable that is used more than once, and shows how this method
correctly rejects it.

\begin{code}
letrec f = \case
         True -> f False + f False
         False -> x
    in f True
\end{code}

To compute the usage environment of $f$ we take its naive environment to be $x
:= 1, f := 2$. Then, we compute the final environment to be $x := 1$ scaled by
the usage of $f$, $2*[x := 1]$, resulting in $x := 2$. Now, to lint the
linearity of the whole let we must ensure the body of the let uses $x$ linearly.
We emit from the body the usage environment of $f$ ($x := 2$) which uses $x$
more than once, i.e. not linearly.

% To compute the usage environment of $f$ we take the second branch usage
% environment $[x := 1]$ and the first branch usage $[rec := 1+1]$ and unify them,
% somehow resulting in $[x := 2]$. Now, to lint the linearity in the whole let
% expression we must ensure that the body of the let uses $x$ linearly. The body
% is $f True$, which is a let-bound variable (how to distinguish functions that
% must be applied vs variables) so we take its usage environment $[x := 2]$ which
% does not use $x$ linearly and thus breaks linearity.

% \textbf{Re-explained:} to compute the usage environment of the recursive
% let-bound function $f$ when applied, we compute $Z$ such that for all branch
% alternatives $U,V,...$, $U \subset Z$, $V \subset Z$ and so on by tracking the
% multiplicities and usage environments of variables that show up in the body and
% by emitting a special keyword $rec$ everytime we find a saturated call of $f$
% (how to handle unsaturated calls?) (how to have a more general solution that
% doesn't require a keyword, perhaps something to do with fixed points); Then we
% scale $Z \setminus \{(rec,_)\}$ by $Z[rec]$ and get $T$ which is the usage
% environment of $f$ when saturated.

% \textbf{Example 1 revisited:}
% \begin{enumerate}
%     \item Take $U = \{rec := 1\}$
%     \item Take $V = \{x := 1\}$
%     \item Take $Z = \{x := 1, rec := 1\}$
%     \item Take $\pi = Z[rec] = 1$
%     \item Take $W = Z \setminus \{rec\} = \{x := 1\}$
%     \item Take $T = \pi W = \{x := \pi * 1\} = \{x := 1\}$
%     \item Linearity OK
% \end{enumerate}

% \textbf{Example 2 revisited:}
% \begin{enumerate}
%     \item Take $U = \{rec := 2\}$
%     \item Take $V = \{x := 1\}$
%     \item Take $Z = \{x := 1, rec := 1\}$
%     \item Take $\pi = Z[rec] = 2$
%     \item Take $W = Z \setminus \{rec\} = \{x := 1\}$
%     \item Take $T = \pi W = \{x := \pi * 1\} = \{x := 2\}$
%     \item Linearity not OK
% \end{enumerate}


% Draft typing rule:

% In haskell that would look like this;
% \begin{code}
% computeRecUsageEnvs :: [(Var, UsageEnv)] -- Recursive usage environments associated to their recursive call
%                     -> [(Var, UsageEnv)] -- Non-recursive usage environments
% computeRecUsageEnvs l =
%   foldl (flip \\(v,vEnv) -> map \\(n, nEnv) -> (n, ((fromMaybe 0 \$ v `M.lookup` nEnv) `scale` (M.delete v vEnv)) `sup` (M.delete v nEnv))) l l

% sup :: UsageEnv -> UsageEnv -> UsageEnv
% sup = M.merge M.preserveMissing M.preserveMissing (M.zipWithMatched \$ \_ x y -> max x y)

% scale :: Mult -> UsageEnv -> UsageEnv
% scale m = M.map (*m)
% \end{code}

\subsection{Handling the case binder\label{casebinder}}

% (though in the rule seems to only start counting after the second usage which
% would make the rule wrong for a case expression that doesn't use the binder --
% the scrutinee should also be consumed)
The current typing rule for case expressions describes that using the case
binder is as using the case scrutinee multiple times, so we scale the usage of
the case scrutinee by the superior multiplicity of using the case binder across
all branches. However, we hypothetize that this prevents us from typing many
valid programs and ultimately does not correctly capture the usage of a
variable and present a seemingly viable alternative.

Firstly, we recall the definition of \emph{consuming a value} from Linear
Haskell as
\begin{itemize}
    \item Consuming an atomic value is forcing it to Normal Form (NF)
    \item Consuming a value that is constructed with more than 1 argument is
        consuming all of its arguments
\end{itemize}
and further note that when an expression is scrutinized by a case expression it
is evaluated to Weak Head Normal Form (WHNF), which in the case of a nullary data
constructor is equal to NF.

Now, with the following example, my interpretation of the current rule dictates
that the case expression consumes the scrutinee twice because the case binder is
used. However, we know that in the branch where $z$ is used, $z$ is equal to
False, and using False does not violate linearity since we can unrestrictedly
create nullary data constructors. A situation similar to this one below happens
after the CSE transformation.
\begin{code}
    case <complex expression> of z {
        False -> case y of z' { DEFAULT -> z }
        True  -> y
    }
\end{code}
% $\leadsto U$ 

A different example is using the case binder $z$ instead of the arguments we
pattern matched on $x,y$. This currently violates linearity because both $x$ and
$y$ aren't used and because $z$ is used twice. This doesn't typecheck in the
frontend either. However, we know that in this branch, using $z$ is effectively
the same as using $(x,y)$!
\begin{code}
    case <complex expression> of z {
        (x,y) -> z
    }
\end{code}
% $\leadsto U$ 

The good news is that both these programs (and a handful of others listed in
Section~\ref{examples}) are accepted with a new rule we have devised for the
linear linter. We still need to prove we don't accept any invalid programs as
valid.

The key idea of the case-binder-usage solution is \textbf{annotating the case binder
with independent usage environments for each pattern match}. That is, for
each of the possible branches, using the case binder $z$ will equate to
using its usage environment in that branch. That makes it so that using the
case binder instead of a nullary data constructor is the same, using the case
binder instead of reconstructing the value with the bound pattern variables too,
and other situations in which we previously couldn't typecheck linearity are
now accepted.

The typing rule for the case expression using the case binder solution:

\begin{mathparpagebreakable}
    \infer*[right=(case)]
    {\Gamma \vdash t : D_{\pi_1 \dots \pi_n} \leadsto \{U\} \\
     \Gamma ; z :_{U_k} D_{\pi_1 \dots \pi_n} \vdash b_k : C \leadsto \{V_k\}
     \and V_k \leq V}
    {\Gamma \vdash \text{case } t \text{ of } z :_{(U_1\dots U_n)} D_{\pi_1 \dots \pi_n} \{b_k\}_1^m : C \leadsto \{U + V\}}
\end{mathparpagebreakable}

Taking the first example in this section, we annotate both branches (True and
False) with $U_z = \emptyset$ because the pattern match doesn't introduce any
new variables. We consume \emph{complex expression} once and $y$ once, and then use
$z$ once. However, using $z$ once equates to not using anything (semantically
this relates to being able to use False or True unrestrictedly) and therefore
linearity is preserved.

In the second example, we know annotate the only branch with usage environment
$U_z = [x := 1, y := 1]$. Upon the usage of $z$ we emit $x$ and $y$ --
effectively using both -- thus linearity is preserved.

\section{Multiplicity Coercions\label{multiplicityCoercions}}

The second set of problems arises from our inability to coerce a multiplicity
into another (or say that one is submultiple of another?).

% When we pattern match on a GADT we ...

Taking the following example we can see that we don't know how to say
that x is indeed linear in one case and unrestricted in the other, even though
it is according to its type. We'd need some sort of coercion to coerce through
the multiplicity to the new one we uncover when we pattern match on the GADT
evidence (...)

\begin{code}
data SBool :: Bool -> Type where
  STrue :: SBool True
  SFalse :: SBool False

type family If b t f where
  If True t _ = t
  If False _ f = f

dep :: SBool b -> Int %(If b One Many) -> Int
dep STrue x = x
dep SFalse _ = 0
\end{code}


% \section{Novel rules}

% \subsection{The usage environment in a Recursive Let}

% \subsection{The usage environment of a Case Binder\label{caseBinderKey}}


% \begin{mathparpagebreakable}
%     \infer*[right=(letrec)]
%     {\Gamma ; \Delta/ \Delta' ; \Omega \vdash M : A \Uparrow \and \Gamma ;
%     \Delta/ \Delta'' ; \Omega \vdash N : B \Uparrow \and \Delta' = \Delta''}
%     {\Gamma ; \Delta/\Delta' ; \Omega \vdash  (M \with N) : A \leadsto U}
% \end{mathparpagebreakable}


\section{Examples\label{examples}}

In this section we present worked examples of programs that currently fail to
compile with \textbf{-dlinear-core-lint} but would succeed according to the
novel typing rules we introduced above. Each example belongs to a subsection
that indicates after which transformation the linting failed.

\subsection{After the desugarer (before optimizations)}

The definition for $\$!$ in \textbf{linear-base}\cite{linearbase} fails to lint after the
desugarer (before any optimisation takes place) with \emph{Linearity failure in
lambda: x 'Many $\not\subseteq$ p}
\begin{code}
($!) :: forall {rep} a (b :: TYPE rep) p q. (a %p -> b) %q -> a %p -> b
($!) f !x = f x
\end{code}

The desugared version of that function follows below. It violates (naive?)
linearity by using $x$ twice, once to force the value and a second time to call
$f$. However, the $x$ passed as an argument is actually the case binder and it
must be handled in its own way. The case binder is the key (as seen
in~\ref{casebinder}) to solving this
and many other examples.
\begin{code}
($!)
  :: forall {rep :: RuntimeRep} a (b :: TYPE rep) (p :: Multiplicity)
            (q :: Multiplicity).
     (a %p -> b) %q -> a %p -> b
($!)
  = \ (@(rep :: RuntimeRep))
      (@a)
      (@(b :: TYPE rep))
      (@(p :: Multiplicity))
      (@(q :: Multiplicity))
      (f :: a %p -> b)
      (x :: a) ->
      case x of x { __DEFAULT -> f x }
\end{code}

% Intuitively, this program is linear because forcing the polymorphic value to
% WHNF ...? Need better semantics of consuming
%
The case binder usage rule typechecks this program because $x$ is consumed once,
the usage environment of the case binder is empty ($x :_\emptyset a$) and,
therefore, when $x$ is used in $f(x)$, we use its usage environment which is
empty, so we don't use anything new.

\emph{How are strictness annotations typed in the frontend? The issue is this program
consumes to value once to force it, and then again to determine the return
value.}

% Work better the meaning of consumption (does it mean for a value to be reduced
% to WHNF?). Is consuming = forcing? Why is the above program linear?

As a finishing note on this example, we show the resulting program from
\textbf{-ddump-simpl} that simply uses a different name for the case binder.
\begin{code}
($!)
  :: forall {rep :: RuntimeRep} a (b :: TYPE rep)
            (p :: Multiplicity) (q :: Multiplicity).
     (a %p -> b) %q -> a %p -> b
($!)
  = \ (@(rep :: RuntimeRep))
      (@a)
      (@(b :: TYPE rep_aSJ))
      (@(p :: Multiplicity))
      (@(q :: Multiplicity))
      (f :: a %p -> b)
      (x :: a) ->
      case x of y { __DEFAULT -> f_aDV y }
\end{code}
% So in this program we use another name for the case binder, but it still stands
% for the resulting of evaluating to WHNF; how is it technically different?

\subsection{Common Sub-expression Elimination}

Currently, the CSE seems to transform a linear program that pattern matches on
constant and returns the same constant into a program that breaks linearity that
pattern matches on the argument and returns the argument (where in the frontend
a constant equal to the argument would be returned)

The definition for $\&\&$ in \textbf{linear-base} fails to lint after the common
sub-expression elimination (CSE) pass transforms the program.
\begin{code}
(&&) :: Bool %1 -> Bool %1 -> Bool
False && False = False
False && True = False
True && x = x
\end{code}
The issue with the program resulting from the transformation is $x$ being used
twice in the first branch of the first case expression. We pattern match on y to
force it (since it's a type without constructor arguments, forcing is consuming)
and then return $x$ rather than the constant $False$
\begin{code}
(&&) :: Bool %1 -> Bool %1 -> Bool
(&&) = \ (x :: Bool) (y :: Bool) ->
  case x of w1 {
    False -> case y of w2 { __DEFAULT -> x };
    True -> y
  }
\end{code}
At a first glance, the resulting program is impossible to typecheck linearly.

The key observation is that after $x$ is forced into either $True$ or $False$,
we know $x$ to be a constructor without arguments (which can be created
without restrictions) and know that where we see $x$ we can as well have
$True$ or $False$ depending on the branch. However, using $x$ here is very
unsatisfactory (and linearly unsound?) because $x$ might be an expression, and
we can't really associate $x$ to the value we pattern matched on (be it $True$
or $False$). What we really want to have instead of $x$ is the $w1$ case binder --
it relates directly to the value we pattern matched on, it's a variable rather
than an expression, and, most importantly, our case-binder-usage idea could be
applied here as well

% Idea: if $x$ was anotated with some information regarding the CSE then perhaps
% it could be typechecked (in a system that considered said anotations) -- This is
% too specific and the case binder environment solution is much cleaner. It just
% requires that the example is changed into using the case binder $w1$ rather than
% $x$.

The solution with the unifying case binder usage idea means annotating the case
binder with its usage environment. For $True$ and $False$ (and other data
constructors without arguments) the usage environment is always empty (using
them doesn't entail using any other variable), meaning we can always use the
case binder instead of the actual constant constructor without issues.
%
Concretely, if we had $w1$ instead of the second occurrence of $x$, we'd have
an empty usage environment for $w1$ in the $False$ branch ($w1 :_\emptyset
Bool$) and upon using $w1$ we wouldn't use any extra resources


To make this work, we'd have to look at the current transformations and see
whether it would be possible to ensure that CSEing the case scrutinee would
entail using the case binder instead of the scrutinee. I don't know of a
situation in which we'd really want the scrutinee rather than the case binder,
so I hypothetize the modified transformation would work out in practice and
solve this linearity issue.


% In this exact example, the binder has usage environment empty ($w1 := []$),
% meaning that in the False branch, when we use $x$, if we instead used $w1$ which
% is equivalent and seems more correct (since it doesn't need the case scrutinee
% expression to be a variable), then the usage of $w1$ would imply the usage of
% $[]$ which is nothing and therefore we would preserve linearity

Curiously, the optimised program resulting from running all transformations
actually does what we expected it with regard to using the constant constructors
and preserving linearity. So is the issue from running linear lint after a
particular CSE but it would be fine in the end?
\begin{code}
(&&) :: Bool %1 -> Bool %1 -> Bool
(&&) = \ (x :: Bool) (y :: Bool) ->
  case x of {
    False -> case y of { __DEFAULT -> GHC.Types.False };
    True -> y
  }
\end{code}

\subsection{Compiling ghc with -dlinear-core-lint}

From the definition of $\mathit{groupBy}$ in $\mathit{GHC.Data.List.Infinite}$:

\begin{code}
groupBy :: (a -> a -> Bool) -> Infinite a -> Infinite (NonEmpty a)
groupBy eq = go
  where
    go (Inf a as) = Inf (a:|bs) (go cs)
      where (bs, cs) = span (eq a) as
\end{code}

We get a somewhat large resulting core expression, in the middle of which the
following expression with a linear type -- which currently syntatically violates
linearity.

\begin{code}
jp :: ([a], Infinite a) \%1 -> (# [a], Infinite a #)
jp (wwj :: ([a], Infinite a)) =
    case wwj of wwj {
        DEFAULT -> case wwj of { (wA, wB) -> (# wA, wB #) }
    }
\end{code}

Using the case binder usage solution, the case binder is annotated with a usage
environment for the only branch ($U_wwj = \emptyset$) and using $wwj$ is equal
to using the $\emptyset$. Here it would mean that second \textbf{case of} $wwj$
doesn't actually use any variables and hence the first $wwj$ isn't used twice.


% \begin{code}
% groupBy = \ (@a) (eq :: a -> a -> Bool) (eta :: Infinite a) ->
%   letrec {
%     go :: Infinite a -> Infinite (NonEmpty a)
%     go = \ (inf :: Infinite a) -> case inf of {
%       Inf x xs -> let {
%         ds :: ([a], Infinite a)
%         ds = let {
%             parteq :: a -> Bool
%             parteq = eq x
%         } in
%           letrec {
%             goZ :: Infinite a -> ([a], Infinite a)
%             goZ = \ (inf' :: Infinite a) -> case wgo inf' of { (# wA, wB #) -> (wA, wB) };

%             wgo :: Infinite a -> (# [a], Infinite a #)
%             wgo = \ (inf' :: Infinite a) -> case inf' of wildX2 {
%               Inf y ys ->
%                 join {
%                     jp :: ([a], Infinite a) \%1 -> (# [a], Infinite a #)
%                     jp (wwj :: ([a], Infinite a)) =
%                         case wwj of wwj {
%                             DEFAULT -> case wwj of { (wA, wB) -> (# wA, wB #) } }
%                     } in
%                       case parteq y of {
%                         False ->
%                             let {
%                                 ww :: ([a], Infinite a)
%                                 ww = ([] @a, wildX2)
%                             } in jump jp ww;
%                         True ->
%                             let {
%                                 dy :: ([a], Infinite a)
%                                 dy = case wgo ys of { (# wA, wB #) -> (wA, wB)
%                                 }
%                             } in
%                                 let {
%                                     wwB :: [a]
%                                     wwB = case dy of { (bs, cs) -> bs }
%                                 } in
%                                     let {
%                                         wwA :: [a]
%                                         wwA = : @a y wwB
%                                     } in let {
%                                         wwC :: Infinite a
%                                         wwC = case dy of { (bs, cs) -> cs }
%                                     } in let {
%                                         ww :: ([a], Infinite a)
%                                         ww = (wwA, wwC)
%                                     } in jump jp ww
%                                 }
%               };
%           } in
%               case wgo xs of { (# wA, wB #) -> (wA, wB) }
%       } in
%           Inf @(NonEmpty a) (:| @a x (case ds of { (bs, cs) -> bs })) (case ds of { (bs, cs) -> go cs })
%     };
%   } in go eta
% \end{code}
% The issue is in the linear function that shows up in the core output (how does
% the linearity end up there?). The wwj variable is used once in the case
% expression, bound as the case binder, and used in the case body once again. Why
% do we do that instead of pattern matching right away? Seems a bit redundant.

% Observation: If the branch is DEFAULT, then the case binder binds the case
% scrutinee which was just forced?, but hasn't been actually consumed, because we
% haven't consumed its components. As long as the same branch doesn't consume the
% pattern matching result and the case binder at the same time it should be fine?

% In this example, we force wwj with the case binder, but we don't really
% consume it (more precise definition of consume...), so we can use the case
% binder meaning we're simply using x for the first time. Needs to be
% formalised....


\subsection{Artificial examples}

Following the difficulties in consuming or not consuming the case binder and its
associated bound variables linearly we constructed some additional examples that
bring out the problem quite clearly.
\begin{code}
case x of w1
    (x1,x2) -> case y:Bool of
                DEFAULT -> (x1,x2)
\end{code}
Is equal to (note that the case binder is being used rather than the $x$ case
scrutinee which could be an arbitrary expression and hence really cannot be
consumed multiple times?)
\begin{code}
case x of w1
    (x1,x2) -> case y:Bool of
                DEFAULT -> w1
\end{code}

We initially wondered whether x was being consumed if the pattern match ignored
some of its variables. We pose that if it does have wildcards then it isn't
consuming x fully
\begin{code}
case x of w1
    (_,x2) -> Is x being consumed? no because not all of its components are
    being consumed?
\end{code}

Can we have a solution that even handles weird cases in which we pattern match
twice on the same expression but only on one constructor argument per time? I
don't think so.
\begin{code}
case exp of w1
    (x1,_) -> case w1 of w2
                (_, x2) -> (x1, x2)
\end{code}

A double let rec in which both use a linear bound variable $y$
\begin{code}
letrec f = \case
        True  -> y
        False -> g True
       g = \case
        True -> f True
        False -> y
    in f False
\end{code}

We have to compute the usage environment of f and g.
For f, we have the first branch with usage environment $y$ and the second with
usage $g$, meaning we have $F = \{g := 1, y := 1\}$
For g, we have $G = \{f := 1, y := 1\}$. To calculate the usage environment of
$f$ to emit upon $f False$, we calculate ...

Refer to the algorithm for computing recursive environment


% \section{Work in progress}

% % \section{Core to Core Validation}
% \subsection{Inlining}

% \subsection{CSE}

% \subsection{Join Points}

% When duplicating a case (in the case-of-case transformation), to avoid code
% explosion, the branches of the case are first made into join points

% \begin{code}
% case e of
%   Pat1 -> u
%   Pat2 -> v
% ~~>
% let j1 = u in
% let j2 = v in
% case e of
%   Pat1 -> j1
%   Pat2 -> j2
% \end{code}

% If there is any linear variable in u and v, then the standard
% let rule above will fail (since j1 occurs only in one branch, and
% so does j2).

% However, if j1 and j2 were annotated with their usage environment,

% \subsection{Empty Case}

% For case expressions, the usage environment is computed by checking all branches
% and taking sup. However, this trick doesn't work when there are no branches.

% \begin{itemize}
% \item https://gitlab.haskell.org/ghc/ghc/-/issues/20058
% \item https://gitlab.haskell.org/ghc/ghc/-/issues/18768

% \item (1) Just like a case expression remembers its type (Note [Why does Case have a
% 'Type' field?] in Core.hs), it should remember its usage environment. This data
% should be verified by Lint.

% \item (2) Once this is done, we can remove the Bottom usage and the second field of
% UsageEnv. In this step, we have to infer the correct usage environment for empty
% case in the typechecker.
% \end{itemize}

% \begin{code}
% {-# LANGUAGE LinearTypes, EmptyCase #-}
% module M where

% {-# NOINLINE f #-}
% f :: a %1-> ()
% f x = case () of {}
% \end{code}

% This example is well typed: a function is linear if it consumes its argument
% exactly once when it's consumed exactly once. It seems like the function isn't
% linear since it won't consume x because of the empty case, however, that also
% means f won't be consumed due to the same empty case, thus linearity is
% preserved.

% \begin{code}
% * In the case of empty types (see Note [Bottoming expressions]), say
%   data T
% we do NOT want to replace
%   case (x::T) of Bool {}   -->   error Bool "Inaccessible case"
% because x might raise an exception, and *that*'s what we want to see!
% (#6067 is an example.) To preserve semantics we'd have to say
%    x `seq` error Bool "Inaccessible case"
% but the 'seq' is just such a case, so we are back to square 1.
% \end{code}

% There are three different problems:

% \begin{itemize}
% \item castBottomExpr converts (case x :: T of {}) :: T to x.
% \item Worker/wrapper moves the empty case to a separate binding
% \item CorePrep eliminates empty case, just like point 1 (See -- Eliminate empty
%     case in GHC.CoreToStg.Prep
% \end{itemize}

% With castBottomExpr, we get the example above to
% \begin{code}
%     f = \ @a (x (%1) :: a) -> ()
% \end{code}
% And if we don't, 
% \begin{code}
%     f = \ @a (x (%1) :: a) -> case () of {}
% \end{code}
% And that supposedly if we had a usage environment in the case expression we
% could avoid the error. How is it typed without the transformation in face
% of the bottom? (Even knowing that theoretically it's because of divergence?)

\bibliographystyle{plain}
\bibliography{references}

\end{document}

% TODO:Use glossary
