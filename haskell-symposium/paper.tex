\documentclass[sigplan,screen,review]{acmart}


\begin{document}

\title{Typechecking Linearity in Haskell's Core Intermediate Language}

\author{Rodrigo Mesquita}
\affiliation{
  \institution{Universidade Nova de Lisboa}
  \city{Lisbon}
  \country{Portugal}
}

\author{Bernardo Toninho}
\affiliation{
  \institution{Universidade Nova de Lisboa}
  \city{Lisbon}
  \country{Portugal}
}

\begin{abstract}
Linear types were introduced to Haskell ...
%
However, the current Core type-checker rejects many linearly valid programs
that originate from Core-to-Core optimizing transformations. As such, linearity
typing is effectively disabled, for otherwise disabling optimizations would be
far more devastating.
%
% This dissertation presents an extension to Core's type system that accepts a
The goal of our proposed dissertation is to develop an extension to Core's type
system that accepts a larger amount of programs and verifies that optimizing
transformations applied to well-typed linear Core produce well-typed linear
Core.
%
Our extension will be based on attaching variable \emph{usage environments} to
binders, which augment the type system with more  fine-grained contextual
linearity information, allowing the system to accept programs which seem to
syntactically violate linearity but preserve linear resource usage. We will
also develop a usage environment inference procedure and integrate the
procedure with the type checker.  We will validate our proposal by showing a
range of Core-to-Core transformations can be typed by our system.
\end{abstract}

\maketitle

\section{Introduction}

\end{document}
